\section{选题的背景与意义}

\begin{tcolorbox}
	此部分回答\textbf{为何做}。建议分两部分分别阐述:
\end{tcolorbox}

\subsection{选题的背景}

\begin{tcolorbox}
	用一两段话简要介绍该选题的相关背景(如国家政策、业界环境等),篇幅不宜过长,
	但应言之有物。
\end{tcolorbox}

\subsection{选题的意义}

\begin{tcolorbox}
	用一两段话阐述该选题的意义(可包括理论意义与现实意义,如:现实中遇到了哪些困
	难、该工作可以从哪些方面解决这些困难等)。由此引出所做工作正式为解决这些困
	难,要师出有名,只有有意义的工作才值得去做。
\end{tcolorbox}

\section{研究的基本内容与拟解决的主要问题}

\begin{tcolorbox}
	此部分回答\textbf{做什么}。依然建议分两部分分别阐述:
\end{tcolorbox}

\subsection{研究的基本内容}

\begin{tcolorbox}
	分别描述为完成最终目标需要完成哪些子目标,子目标之间如何关联、组合,最终可以
	完成最终目标,即先总后分、环环相扣(该部分应与任务书中描述的任务一一对应,但
	更加细化)。此处建议分条目分别展开阐述,必要时可加入模块图、架构图或流程图等
	图表并适当加以解释说明。
\end{tcolorbox}
\begin{figure}[h]
	\centering
	\includegraphics[width=.6\textwidth]{images/cuz_logo.png}
	\caption{浙传校徽}
	\label{fig:1}
\end{figure}

\subsection{拟解决的主要问题}

\begin{tcolorbox}
	针对上述研究内容中的重点、难点、或涉及到的主要问题分别进行详细阐述,可适当突
	出其难点难在何处。若有亮点或创新点,也可在此处阐述。
\end{tcolorbox}

\section{研究的方法与措施}

\begin{tcolorbox}
	此部分回答\textbf{如何做}:对应上一节中的研究内容与拟解决的主要问题分别阐述
	如何完成任务与解决问题,详细描述解决思路,包括采用哪些方法、平台、框架、技术
	等。
\end{tcolorbox}

\section{研究的总体安排与进度}

\begin{tcolorbox}
	此部分回答\textbf{何时做}:对应上述研究内容中的每项子工作分别给出大致时间节
	点,必要时可画出甘特图,可与任务书中的进度匹配,但不应照搬,尤其不应包括毕设
	程序性安排。
\end{tcolorbox}

\section{主要参考文献}

\begin{tcolorbox}
	此部分无需加入任何内容,模板会自动根据正文中的引用生成参考文献,故\textbf{要
		确保在正文中适当位置、以适当方式引用适当文献}。引用位置尽量不要出现在标
	点符号后,除非整句引用(不提倡);引用方式可采用作者上标型
	(如:\citet{chen1980zhongguo}提出……)、直接上标型(如:采用了……方法
	\citep{chen2005zhulu})或多个参考文献同时引用(如:这些方法
	\citep{Bohan1928,chu2004tushu,Dubrovin1906,hls2012jinji}),但应注意避免
	同时引用过多;引用数目并无明确要求,但在描述到涉及他人成果时应如实引用。
\end{tcolorbox}