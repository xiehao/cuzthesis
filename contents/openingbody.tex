\begin{tcolorbox}
	\emph{郑重声明:本文档仅给出个人建议,仅供参考。另外,在实际撰写时,凡出现于
		tcolorbox环境中的文本均应连同环境本身一同注释或直接删除,以免影响最终呈
		现效果。}
\end{tcolorbox}

\begin{tcolorbox}
	开题报告本质为可行性分析,应让读者相信选题有意义、工作量与难度合适、可保质保
	量按时完成,因此主要回答\emph{为何做}、\emph{做何事}、\emph{如何
		做}、\emph{何时做}等问题。现分别阐述如下:
\end{tcolorbox}

\section{选题的背景与意义}

\begin{tcolorbox}
	此部分回答\emph{为何做}。建议分两部分分别阐述:
\end{tcolorbox}

\subsection{选题的背景}

\begin{tcolorbox}
	用一段话简要介绍该选题的相关背景(如国家政策、业界环境等),篇幅不宜过长,但
	应言之有物。
\end{tcolorbox}

\subsection{选题的意义}

\begin{tcolorbox}
	用一两段话阐述该选题的意义(可包括理论意义与现实意义,如:现实中遇到了哪些困
	难、该工作可以从哪些方面解决这些困难等)。由此引出所做工作正式为解决这些困
	难,要师出有名,只有有意义的工作才值得去做。
\end{tcolorbox}

\section{研究的基本内容与拟解决的主要问题}

\begin{tcolorbox}
	此部分回答\emph{做何事}。依然建议分两部分分别阐述,但应主要着墨于\emph{自己}
	的工作、仅在必要时稍微提及他人工作,\emph{主次分明、详略得当}:
\end{tcolorbox}

\subsection{研究的基本内容}

\begin{tcolorbox}
	分别描述为完成最终目标需要完成哪些子目标,子目标之间如何关联、组合,进而可以
	完成最终目标,\emph{先总后分、环环相扣}(该部分应与任务书中描述的任务一一对
	应,但更加细化)。此处建议分条目分别展开阐述,必要时可加入模块图、架构图或原
	型示意图等图表并适当加以解释说明。
\end{tcolorbox}
\begin{figure}[h]
	\centering
	\includegraphics[width=.6\textwidth]{images/cuz_logo.png}
	\caption{浙传校徽}
	\label{fig:1}
\end{figure}

\subsection{拟解决的主要问题}

\begin{tcolorbox}
	针对上述研究内容中的重点、难点、或涉及到的主要问题分别进行详细阐述,可适当突
	出其难点难在何处。若有亮点或创新点,也可在此处阐述。
\end{tcolorbox}

\section{研究的方法与措施}

\begin{tcolorbox}
	此部分回答\emph{如何做}:对应上述研究内容与拟解决的主要问题中的每项子工作分
	别阐述如何完成任务与解决问题,详细描述解决思路,给出初步解决方案,论证可行
	性,包括采用哪些方法、平台、框架、技术等,必要时可加入流程图并适当加以解释说
	明。\emph{解决方案的细致程度可从一定程度上反映作者是否已对开展工作做好充分准
		备。}
\end{tcolorbox}

\section{研究的总体安排与进度}

\begin{tcolorbox}
	此部分回答\emph{何时做}:对应上述研究内容与拟解决的主要问题中的每项子工作分
	别给出大致时间节点,必要时可画出甘特图,可与任务书中的进度匹配,但不应照搬,
	尤其不应包括毕设程序性安排。
\end{tcolorbox}

\section{主要参考文献}

\begin{tcolorbox}
	此部分无需加入任何内容,模板会自动根据正文中的引用生成参考文献,故\emph{要确
		保在正文中适当位置、以适当方式引用适当数目的适当文献},具体解释如下:
	\begin{itemize}
		\item 引用\emph{位置}应紧跟在被引内容后,但尽量不要出现在标点符号后,除
		      非整句引用(不提倡);
		\item 引用\emph{方式}可采用作者上标型(如:\citet{chen1980zhongguo}提出
		      ……)、直接上标型(如:采用了……方法\citep{chen2005zhulu})或多个参
		      考文献同时引用(如:这些方法
		      \citep{Bohan1928,chu2004tushu,Dubrovin1906,hls2012jinji}……),但应
		      注意避免同时引用过多;
		\item 引用\emph{数目}并无明确要求,但在描述到涉及他人成果时应如实引用,
		      避免遗漏;
		\item 被引\emph{文献}应尽量选择最新成果(以近3年以内为宜,若确为经典可适
		      当放宽年限)。
	\end{itemize}
\end{tcolorbox}