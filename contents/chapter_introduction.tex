% \chapter{绪论}\label{chap:introduction}
\begin{cuzchapter}{绪论}{chap:introduction}

	\section{课题背景}\label{sec:background}

	考虑到许多同学可能缺乏\LaTeX{}使用经验,cuzthesis在参考了ucasthesis模板的基
	础上,将\LaTeX{}的复杂性高度封装,开放出简单的接口,以便轻易使用。同时,对用
	\LaTeX{}撰写论文的一些主要难题,如制图、制表、文献索引等,进行了详细说明,并
	提供了相应的代码样本,理解了上述问题后,对于初学者而言,使用此模板撰写学位论
	文将不存在实质性的困难。所以,如果你是初学者,请不要直接放弃,因为同样为初学
	者的我,十分明白让\LaTeX{}简单易用的重要性,而这正是cuzthesis与ucasthesis所
	追求和体现的。

	此浙传毕业论文模板cuzthesis基于国科大莫晃锐制作的ucasthesis模板发展而来。当
	前cuzthesis模板满足最新的浙江传媒学院本科毕业论文撰写要求和封面设定,兼顾主
	流操作系统:Windows,Linux,macOS 和主流\LaTeX{}编译引
	擎:\hologo{pdfLaTeX}、 \hologo{XeLaTeX}、\hologo{LuaLaTeX},详细支持情况见
	表\ref{tab:support-status}。支持中文书签、中文渲染、中文粗体显示、拷贝PDF中
	的文本到其他文本编辑器等特性。此外,对模板的文档结构进行了精心设计,撰写了编
	译脚本提高模板的易用性和使用效率。
	\begin{table}[htbp]
		\caption[编译引擎跨平台情况]{各平台下编译引擎支持情况(\checkmark:支持或部分支持;$\times$:不支持)}
		\label{tab:support-status}
		\centering
		\small% fontsize
		% \setlength{\tabcolsep}{4pt}% column separation
		% \renewcommand{\arraystretch}{1.2}%row space 
		\begin{tabular}{cccc}
			\toprule
			                     & \hologo{pdfLaTeX}                          & \hologo{XeLaTeX}                     & \hologo{LuaLaTeX} \\
			\midrule
			Linux                & $\times$                                   & \checkmark\footnote{暂不完全支持,粗楷体加由粗宋体代
			替,仿宋加粗无效;但不影响本模板使用。} & \checkmark\footnote{暂不完全支
			持,粗楷体加由粗宋体代替,仿宋加粗无效;但不影响本模板使用。}                                                                                              \\
			macOS                & $\times$                                   & \checkmark\footnote{暂不完全支持,仿宋加粗无效;但不
			影响本模板使用。}            & $\times$                                                                                              \\
			Windows              & \checkmark\footnote{暂不完全支持,粗宋体加由黑体代替。}     &
			\checkmark\footnote{暂不完全支持,粗楷体由粗宋体代替;但不影响本模板使
			用。}                  & \checkmark\footnote{暂不完全支持,所有中文字体均无法加粗,且编译
			时间较\hologo{XeLaTeX}慢一些。}                                                                                                     \\
			\bottomrule
		\end{tabular}
	\end{table}

	cuzthesis的目标在于简化毕业论文的撰写,利用\LaTeX{}格式与内容分离的特征,模
	板将格式设计好后,作者可只需关注论文内容。同时,cuzthesis有着整洁一致的代码
	结构和扼要的注解,对文档的仔细阅读可为初学者提供一个学习\LaTeX{}的窗口。此
	外,模板的架构十分注重通用性,事实上,与ucasthesis一样,cuzthesis不仅是浙传
	毕业论文模板,同时,通过少量修改即可成为使用\LaTeX{}撰写中英文文章或书籍的通
	用模板,并为使用者的个性化设定提供了接口。

	\section{系统要求}\label{sec:system}

	\href{https://github.com/xiehao/CUZThesis}{cuzthesis}宏包可以在目前主流的
	\href{https://en.wikibooks.org/wiki/LaTeX/Introduction}{\LaTeX{}}编译系统中
	使用,例如C\TeX{}套装(请勿混淆C\TeX{}套装与ctex宏包。C\TeX{}套装是集成了许
	多\LaTeX{}组件的\LaTeX{}编译系统,因已停止维护,\textbf{不再建议使用}。
	\href{https://ctan.org/pkg/ctex?lang=en}{ctex} 宏包如同cuzthesis,是\LaTeX{}
	命令集,其维护状态活跃,并被主流的\LaTeX{}编译系统默认集成,是几乎所有
	\LaTeX{}中文文档的核心架构。)、MiK\TeX{}(维护较不稳定,\textbf{不太推荐使
	用})、\TeX{}Live。而文本编辑器方面则包括:Visual Studio Code(简称VS
	Code)、\hologo{TeX}studio、 Emacs、Vim等。推荐的
	\href{https://en.wikibooks.org/wiki/LaTeX/Installation}{\LaTeX{}编译系统}和
	\href{https://en.wikibooks.org/wiki/LaTeX/Installation}{\LaTeX{}文本编辑器}
	如表\ref{tab:recomendations}所示。
	\begin{table}[htbp]
		\caption[推荐的编译系统与编辑器]{不同平台下推荐的编译系统与编辑器}
		\label{tab:recomendations}
		\centering
		\small% fontsize
		%\setlength{\tabcolsep}{4pt}% column separation
		%\renewcommand{\arraystretch}{1.5}% row space 
		\begin{tabular}{ccc}
			\toprule
			%\multicolumn{num_of_cols_to_merge}{alignment}{contents} \\
			%\cline{i-j}% partial hline from column i to column j
			操作系统    & \LaTeX{}编译系统                                            & \LaTeX{}文本编辑器 \\
			\midrule
			Linux   &
			\href{https://www.tug.org/texlive/acquire-netinstall.html}{\TeX{}Live
			Full}   & \href{https://code.visualstudio.com/download}{VS Code}、
			\href{https://www.texstudio.org/}{\hologo{TeX}studio}、Emacs、Vim                   \\
			MacOS   & \href{https://www.tug.org/mactex/}{Mac\TeX{} Full}      &
			\href{https://code.visualstudio.com/download}{VS Code}、
			\href{https://www.texstudio.org/}{\hologo{TeX}studio}、Emacs                       \\
			Windows &
			\href{https://www.tug.org/texlive/acquire-netinstall.html}{\TeX{}Live
			Full}   & \href{https://code.visualstudio.com/download}{VS Code}、
			\href{https://www.texstudio.org/}{\hologo{TeX}studio}                             \\
			\bottomrule
		\end{tabular}
	\end{table}

	\LaTeX{}编译系统,如\TeX{}Live(Mac\TeX{}为针对macOS的\TeX{}Live),用于提供
	编译环境;\LaTeX{}文本编辑器 (如VS Code) 用于编辑\TeX{}源文件。请从各软件官
	网下载安装程序,勿使用不明程序源。\textbf{\LaTeX{}编译系统和\LaTeX{}编辑器分
	别安装成功后,即完成了\LaTeX{}的系统配置},无需其他手动干预和配置。若系统原
	带有旧版的\LaTeX{}编译系统并想安装新版,请\textbf{先卸载干净旧版再安装新
	版}。

	\section{问题反馈}\label{sec:callback}

	关于\LaTeX{}的知识性问题,请查阅
	\href{https://github.com/mohuangrui/ucasthesis/wiki}{ucasthesis和\LaTeX{}知
	识小站} 和 \href{https://en.wikibooks.org/wiki/LaTeX}{\LaTeX{} Wikibook}。

	关于模板编译和样式设计的问题,请\textbf{先仔细阅读此说明文档,特别是“常见问
		题” (章节~\ref{sec:qa})}。若问题仍无法得到解决,请\textbf{先将问题理解
		清楚并描述清楚,再将问题反馈}至
		\href{https://github.com/xiehao/CUZThesis/issues}{Github/cuzthesis/issues}。

	欢迎大家有效地反馈模板不足之处,一起不断改进模板。希望大家向同事积极推广
	\LaTeX{},一起更高效地做科研。

	\section{模板下载}\label{sec:download}

	\begin{center}
		\href{https://github.com/xiehao/CUZThesis}{Github/cuzthesis}:
		\url{https://github.com/xiehao/CUZThesis}。
	\end{center}

\end{cuzchapter}