%---------------------------------------------------------------------------%
%-                                                                         -%
%-                           中文摘要                                      -%
%-                                                                         -%
%---------------------------------------------------------------------------%
% 摘要是论文的重要组成部分,是对论文研究内容和成果的高度概括
% 摘要应包括以下几个方面:
% 1. 研究目的和背景
% 2. 研究方法和过程
% 3. 研究结果和结论
% 4. 研究的创新点和意义

\begin{chineseabstract}
	{浙江传媒学院;毕业论文;\hologo{LaTeX}模板;排版工具;学术写作}本文是浙江传媒学院本
	科毕业论文模板cuzthesis的使用说明文档。主要内容为介绍\hologo{LaTeX}文档类
	cuzthesis的用法,以及撰写毕业论文的一些建议和指导。

	\begin{leftbar}
		\noindent\textbf{建议:}摘要是全文的浓缩,要涵盖文章的主要内容,一般应保
		证字数在300以上,须包括如下三点:做了什么、如何做的、做得怎样。具体而
		言:\textbf{第一段},可以用一句话概述课题背景;接着再用一两句话表明做的
		东西。\textbf{第二段},可以用三五句话描述如何做的,包括做这个东西分几步
		(或几个模块),每步(模块)的任务,以及相邻步(或相关模块)之间的关系。
		\textbf{最后一段},用两三句话概括作品的亮点,以及效果怎样,包括可能的市
		场反响、用户体验等等。

		\noindent{}需要注意的是,既然是全文的浓缩,就不要出现类似“本项目将要做
		到”或者“详情见正文”等此类语句,因其并非正文中应出现的说法(既然已经是正
		文的浓缩,再说详情见正文就产生了矛盾)。此外,语言上应注意言简意赅,语气
		上应平实无感。
	\end{leftbar}
\end{chineseabstract}
%---------------------------------------------------------------------------%
%-                                                                         -%
%-                           英文摘要                                      -%
%-                                                                         -%
%---------------------------------------------------------------------------%
% The abstract is an important part of the thesis, providing a high-level summary
% of the research content and results. It should include:
% 1. Research purpose and background
% 2. Research methods and process
% 3. Research results and conclusions
% 4. Innovation points and significance of the research

\begin{englishabstract}
	{Communication University of Zhejiang (CUZ); Thesis; \LaTeX{} Template;
		Typesetting Tool; Academic Writing} This documentation serves as a comprehensive
	guide for the \LaTeX{} class \texttt{cuzthesis}, which is a specialized thesis
	template for the Communication University of Zhejiang. The main content covers
	how to effectively use the \texttt{cuzthesis} template, along with valuable
	suggestions and guidance for writing an academic thesis.

	\begin{leftbar}
		\noindent\textbf{Suggestions:} Abstract is the abstraction of the whole
		thesis, which should cover the main aspects of the thesis, and should
		contain no less than 300 words, involving 3 aspects: what is done, how
		to do it, and what about the result. Specifically, \textbf{1st
			paragraph}, introduce the background by 1 sentence, and what is done by
		1 or 2 sentence(s). \textbf{2nd paragraph}, describe how to do it by 3
		to 5 sentences, including how many steps (modules) are used, what each
		step (module) does, and the relationship between adjacent steps (related
		modules). \textbf{Last paragraph}, extract the most important features
		of the work and its effect by 2 or 3 sentences, including possible
		responses of markets, user experiences, etc..

		\noindent Note that, since it is the abstraction of the whole thesis,
		expressions such as ``this project will'' or ``readers are referred to
		the body of thesis for details'' should be avoided, for this saying
		should not appear in the body of thesis (now that this is the
		abstraction, a refering to the body of thesis generates a conflict).
		Besides, the language should be neat, and the tone should be plain and
		without emotions.
	\end{leftbar}
\end{englishabstract}