% \chapter{绪论}\label{chap:introduction}
\begin{cuzchapter}{绪论}{chap:introduction}

	%---------------------------------------------------------------------------%
	%->> 绪论写作指南
	%---------------------------------------------------------------------------%
	% 绪论是论文的第一章,主要介绍研究的背景、意义、目的、内容和方法等
	% 一个完整的绪论应包含以下几个部分:
	% 1. 研究背景:介绍研究领域的发展历程和现状,指出存在的问题
	% 2. 研究意义:阐述开展本研究的理论意义和实践价值
	% 3. 研究目的:明确说明本研究要解决的具体问题
	% 4. 研究内容:概述论文的主要研究内容和章节安排
	% 5. 研究方法:简要介绍本研究采用的主要研究方法

	\section{研究背景}\label{sec:background}

	随着信息技术的快速发展和学术交流的日益国际化,高质量的学术论文写作和排版变得越来越重要。在当前的学术环境中,一篇内容充实、格式规范的学术论文不仅能够更好地传达研究成果,还能提高研究工作的专业性和可信度。然而,许多学生在论文写作过程中常常面临两大挑战:一是如何组织和表达研究内容,二是如何处理复杂的排版格式要求。

	传统的文字处理软件(如Microsoft Word)虽然使用广泛,但在处理大型学术文档时存在诸多局限性,如格式不一致、公式排版困难、参考文献管理复杂等问题。相比之下,\LaTeX{}作为一种专业的排版系统,以其卓越的数学公式处理能力、自动化的参考文献管理、精确的版面控制和内容与格式分离的特性,在学术界获得了广泛应用。然而,\LaTeX{}的学习曲线较陡峭,许多初学者往往因其复杂性而望而却步。

	考虑到许多同学可能缺乏\LaTeX{}使用经验,cuzthesis在参考了ucasthesis模板的基础上,将\LaTeX{}的复杂性高度封装,开放出简单的接口,以便轻易使用。同时,对用\LaTeX{}撰写论文的一些主要难题,如制图、制表、文献索引等,进行了详细说明,并提供了相应的代码样本,理解了上述问题后,对于初学者而言,使用此模板撰写学位论文将不存在实质性的困难。

	此浙传毕业论文模板cuzthesis基于国科大莫晃锐制作的ucasthesis模板发展而来。当前cuzthesis模板满足最新的浙江传媒学院本科毕业论文撰写要求和封面设定,兼顾主流操作系统:Windows,Linux,macOS 和主流\LaTeX{}编译引擎:\hologo{pdfLaTeX}、 \hologo{XeLaTeX}、\hologo{LuaLaTeX},详细支持情况见表\ref{tab:support-status}。支持中文书签、中文渲染、中文粗体显示、拷贝PDF中的文本到其他文本编辑器等特性。此外,对模板的文档结构进行了精心设计,撰写了编译脚本提高模板的易用性和使用效率。

	\section{研究意义}\label{sec:significance}

	开发和完善浙江传媒学院毕业论文\LaTeX{}模板具有重要的理论和实践意义:

	\subsection{理论意义}

	\begin{enumerate}
		\item \textbf{促进学术规范化}:统一的论文模板有助于规范学术写作格式,提高学术成果的专业性和可读性,促进学术交流的标准化。

		\item \textbf{推动技术教育创新}:引导学生使用\LaTeX{}等专业工具进行学术写作,有助于培养学生的技术素养和创新能力,拓展其专业技能范围。

		\item \textbf{支持内容与形式分离}:\LaTeX{}的"内容与形式分离"理念有助于学生更加专注于研究内容本身,而非繁琐的排版细节,从而提高研究质量。
	\end{enumerate}

	\subsection{实践意义}

	\begin{enumerate}
		\item \textbf{提高论文质量}:规范化的模板和专业的排版工具可以显著提高毕业论文的整体质量和美观度,减少格式错误。

		\item \textbf{节省时间和精力}:自动化的参考文献管理、章节编号、目录生成等功能可以大大减少学生在排版上花费的时间和精力。

		\item \textbf{增强学术竞争力}:掌握\LaTeX{}等专业工具可以增强学生的学术竞争力,为其未来的学术研究或职业发展奠定基础。

		\item \textbf{促进学校学术形象}:统一规范的高质量论文格式有助于提升学校的学术形象和声誉。
	\end{enumerate}
	\begin{table}[htbp]
		\caption[编译引擎跨平台情况]{各平台下编译引擎支持情况(\checkmark:支持或部分支持;$\times$:不支持)}
		\label{tab:support-status}
		\centering
		\small% fontsize
		% \setlength{\tabcolsep}{4pt}% column separation
		% \renewcommand{\arraystretch}{1.2}%row space
		\begin{tabular}{cccc}
			\toprule
			                     & \hologo{pdfLaTeX}                          & \hologo{XeLaTeX}                     & \hologo{LuaLaTeX} \\
			\midrule
			Linux                & $\times$                                   & \checkmark\footnote{暂不完全支持,粗楷体加由粗宋体代
			替,仿宋加粗无效;但不影响本模板使用。} & \checkmark\footnote{暂不完全支
			持,粗楷体加由粗宋体代替,仿宋加粗无效;但不影响本模板使用。}                                                                                              \\
			macOS                & $\times$                                   & \checkmark\footnote{暂不完全支持,仿宋加粗无效;但不
			影响本模板使用。}            & $\times$                                                                                              \\
			Windows              & \checkmark\footnote{暂不完全支持,粗宋体加由黑体代替。}     &
			\checkmark\footnote{暂不完全支持,粗楷体由粗宋体代替;但不影响本模板使
			用。}                  & \checkmark\footnote{暂不完全支持,所有中文字体均无法加粗,且编译
			时间较\hologo{XeLaTeX}慢一些。}                                                                                                     \\
			\bottomrule
		\end{tabular}
	\end{table}

	\section{研究目的}\label{sec:purpose}

	本研究旨在开发一个全面、易用且符合学术标准的浙江传媒学院毕业论文\LaTeX{}模板,具体目的包括:

	\begin{enumerate}
		\item 创建一个符合浙江传媒学院最新毕业论文格式要求的\LaTeX{}模板,确保学生提交的论文格式规范统一。

		\item 简化\LaTeX{}的使用流程,降低学习门槛,使初学者能够快速掌握模板的使用方法。

		\item 提供全面的论文写作指导,帮助学生理解学术论文的结构和内容要求,提高论文质量。

		\item 构建一个可扩展的模板框架,便于未来根据学校要求的变化进行更新和维护。

		\item 推广\LaTeX{}在学术写作中的应用,提高学生的学术写作能力和技术素养。
	\end{enumerate}

	\section{研究内容}\label{sec:content}

	本研究的主要内容包括以下几个方面:

	\subsection{模板设计与开发}

	cuzthesis的目标在于简化毕业论文的撰写,利用\LaTeX{}格式与内容分离的特征,模板将格式设计好后,作者可只需关注论文内容。同时,cuzthesis有着整洁一致的代码结构和扼要的注解,对文档的仔细阅读可为初学者提供一个学习\LaTeX{}的窗口。此外,模板的架构十分注重通用性,事实上,与ucasthesis一样,cuzthesis不仅是浙传毕业论文模板,同时,通过少量修改即可成为使用\LaTeX{}撰写中英文文章或书籍的通用模板,并为使用者的个性化设定提供了接口。

	\subsection{论文写作指导}

	本研究不仅提供了\LaTeX{}模板,还包含了全面的论文写作指导,涵盖以下内容:

	\begin{itemize}
		\item \textbf{论文结构指导}:详细介绍学术论文的标准结构,包括绪论、文献综述、研究方法、结果分析、讨论和结论等部分的写作要点。

		\item \textbf{学术写作规范}:提供学术写作的基本规范,包括语言表达、引用格式、图表制作等方面的指导。

		\item \textbf{常见问题解答}:针对论文写作和\LaTeX{}使用过程中的常见问题提供解答和建议。
	\end{itemize}

	\subsection{使用示例与案例}

	为了帮助用户更好地理解和使用模板,本研究提供了丰富的使用示例和案例,包括:

	\begin{itemize}
		\item 各类型内容(文本、公式、图表、代码等)的排版示例
		\item 不同学科论文的格式案例
		\item 常见排版问题的解决方案
	\end{itemize}

	\section{研究方法}\label{sec:method}

	本研究采用以下方法进行模板的设计、开发和完善:

	\subsection{文献研究法}

	通过查阅国内外相关文献,了解\LaTeX{}模板设计的最新进展和最佳实践,为模板开发提供理论基础。同时,研究浙江传媒学院的论文格式要求和学术规范,确保模板符合学校标准。

	\subsection{比较分析法}

	对比分析国内外知名高校的\LaTeX{}论文模板,如清华大学、北京大学、中国科学院大学等机构的模板,吸取其优点,避免其不足,为模板设计提供参考。

	\subsection{实验开发法}

	采用迭代开发的方式,不断测试和完善模板功能,确保模板的稳定性和易用性。在开发过程中,注重用户体验,简化操作流程,降低学习门槛。

	\subsection{用户反馈法}

	收集和分析用户反馈,了解用户在使用过程中遇到的问题和需求,及时进行修复和改进,不断提高模板的质量和用户满意度。

	\section{系统要求}\label{sec:system}

	\href{https://github.com/xiehao/CUZThesis}{cuzthesis}宏包可以在目前主流的
	\href{https://en.wikibooks.org/wiki/LaTeX/Introduction}{\LaTeX{}}编译系统中
	使用,例如C\TeX{}套装(请勿混淆C\TeX{}套装与ctex宏包。C\TeX{}套装是集成了许
	多\LaTeX{}组件的\LaTeX{}编译系统,因已停止维护,\textbf{不再建议使用}。
	\href{https://ctan.org/pkg/ctex?lang=en}{ctex} 宏包如同cuzthesis,是\LaTeX{}
	命令集,其维护状态活跃,并被主流的\LaTeX{}编译系统默认集成,是几乎所有
	\LaTeX{}中文文档的核心架构。)、MiK\TeX{}(维护较不稳定,\textbf{不太推荐使
	用})、\TeX{}Live。而文本编辑器方面则包括:Visual Studio Code(简称VS
	Code)、\hologo{TeX}studio、 Emacs、Vim等。推荐的
	\href{https://en.wikibooks.org/wiki/LaTeX/Installation}{\LaTeX{}编译系统}和
	\href{https://en.wikibooks.org/wiki/LaTeX/Installation}{\LaTeX{}文本编辑器}
	如表\ref{tab:recomendations}所示。
	\begin{table}[htbp]
		\caption[推荐的编译系统与编辑器]{不同平台下推荐的编译系统与编辑器}
		\label{tab:recomendations}
		\centering
		\small% fontsize
		%\setlength{\tabcolsep}{4pt}% column separation
		%\renewcommand{\arraystretch}{1.5}% row space
		\begin{tabular}{ccc}
			\toprule
			%\multicolumn{num_of_cols_to_merge}{alignment}{contents} \\
			%\cline{i-j}% partial hline from column i to column j
			操作系统    & \LaTeX{}编译系统                                            & \LaTeX{}文本编辑器 \\
			\midrule
			Linux   &
			\href{https://www.tug.org/texlive/acquire-netinstall.html}{\TeX{}Live
			Full}   & \href{https://code.visualstudio.com/download}{VS Code}、
			\href{https://www.texstudio.org/}{\hologo{TeX}studio}、Emacs、Vim                   \\
			MacOS   & \href{https://www.tug.org/mactex/}{Mac\TeX{} Full}      &
			\href{https://code.visualstudio.com/download}{VS Code}、
			\href{https://www.texstudio.org/}{\hologo{TeX}studio}、Emacs                       \\
			Windows &
			\href{https://www.tug.org/texlive/acquire-netinstall.html}{\TeX{}Live
			Full}   & \href{https://code.visualstudio.com/download}{VS Code}、
			\href{https://www.texstudio.org/}{\hologo{TeX}studio}                             \\
			\bottomrule
		\end{tabular}
	\end{table}

	\LaTeX{}编译系统,如\TeX{}Live(Mac\TeX{}为针对macOS的\TeX{}Live),用于提供
	编译环境;\LaTeX{}文本编辑器 (如VS Code) 用于编辑\TeX{}源文件。请从各软件官
	网下载安装程序,勿使用不明程序源。\textbf{\LaTeX{}编译系统和\LaTeX{}编辑器分
	别安装成功后,即完成了\LaTeX{}的系统配置},无需其他手动干预和配置。若系统原
	带有旧版的\LaTeX{}编译系统并想安装新版,请\textbf{先卸载干净旧版再安装新
	版}。

	\section{论文结构}\label{sec:structure}

	本论文共分为五章,各章内容概述如下:

	\textbf{第一章 绪论}:介绍研究背景、研究意义、研究目的、研究内容和研究方法,阐述开发浙江传媒学院毕业论文\LaTeX{}模板的必要性和重要性,并概述系统要求和使用环境。

	\textbf{第二章 文献综述}:回顾和评述国内外相关研究成果,包括\LaTeX{}在学术写作中的应用、国内外高校论文模板的发展现状以及学术论文写作规范等方面的研究,为模板开发提供理论基础。

	\textbf{第三章 模板设计与实现}:详细介绍模板的设计思路、架构和实现方法,包括文档类的设计、页面布局、字体设置、章节样式、图表格式、参考文献管理等方面的技术实现。

	\textbf{第四章 使用指南与示例}:提供全面的模板使用指南,包括环境配置、基本用法、高级功能和常见问题解决方案,并通过丰富的示例展示模板的各种功能和应用场景。

	\textbf{第五章 总结与展望}:总结模板开发的主要成果和特点,分析存在的不足,并对未来的改进方向和发展前景进行展望。

	此外,论文还包括参考文献、致谢和附录等部分,提供补充信息和资源。



	\section{问题反馈}\label{sec:callback}

	关于\LaTeX{}的知识性问题,请查阅
	\href{https://github.com/mohuangrui/ucasthesis/wiki}{ucasthesis和\LaTeX{}知
	识小站} 和 \href{https://en.wikibooks.org/wiki/LaTeX}{\LaTeX{} Wikibook}。

	关于模板编译和样式设计的问题,请\textbf{先仔细阅读此说明文档,特别是“常见问
		题” (章节~\ref{sec:qa})}。若问题仍无法得到解决,请\textbf{先将问题理解
		清楚并描述清楚,再将问题反馈}至
		\href{https://github.com/xiehao/CUZThesis/issues}{Github/cuzthesis/issues}。

	欢迎大家有效地反馈模板不足之处,一起不断改进模板。希望大家向同事积极推广
	\LaTeX{},一起更高效地做科研。

	\section{模板下载}\label{sec:download}

	\begin{center}
		\href{https://github.com/xiehao/CUZThesis}{Github/cuzthesis}:
		\url{https://github.com/xiehao/CUZThesis}。
	\end{center}

\end{cuzchapter}