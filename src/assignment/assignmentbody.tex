\section{毕业设计题目}

\makeatletter
\@title % title macro of thesis, DO NOT MODIFY IT!
\makeatother

\section{主要任务与目标}

\begin{tcolorbox}
	首先,用一两句话简要描述该毕设需完成的主要任务(要做什么);其次,列出该任务
	所要达成的目标、预期效果等(做到怎样)。例如:
\end{tcolorbox}

本次毕设课题主要任务为:利用微软的WPF(Windows Presentation Foundation)与.NET框
架,打造一款跨平台桌面任务管理应用,该应用致力于为用户的生活和办公带来便捷。现将
其应达到的目标与效果简要描述如下:
\begin{enumerate}
	\item 具备简要记录并存储备忘录功能;
	\item 具备添加删除待办事项功能;
	\item 具备设置日程并在适当时机提醒用户的功能;
	\item 用户界面应简洁、美观,交互人性化。
\end{enumerate}

\section{主要内容与基本要求}

\begin{tcolorbox}
	首先,展开描述上述任务的主要内容(必要时可画出模块关系图);其次,列出为完成
	上述内容应做到满足哪些基本要求(可在此处引用主要参考文献)。例如:
\end{tcolorbox}

在本次毕设课题作品中,建议前端基于WPF与Prism框架,以C\#作为主要编程语言,结合
XAML 语言呈现前端控件效果;后端需构建SQL数据库,可借助阿里云免费服务器实现云端存
储。具体要求如下:
\begin{enumerate}
	\item 阅读WPF与.NET框架的相关文献
	      \cite{chen1980zhongguo,chen2005zhulu,chu2004tushu,yuan2012lanc,lamport1986document,niu2013zonghe,Bohan1928,Dubrovin1906,hls2012jinji,wikibook2014latex}、
	      整理成文献综述,并学习相关技术;
	\item 尽量设计合理的后端数据库表结构,保障数据存储的高效性;
	\item 尽量合理利用设计模式(如MVVM模式),保证代码结构清晰、模块划分合理,提
	      升代码的可读性与可维护性,便于后期维护与扩展;
	\item 遇到问题应及时与指导教师沟通,尽量按进度完成任务。
\end{enumerate}

\section{毕业设计进度安排}

\begin{tcolorbox}
	按学校要求的时间节点合理安排进度即可,例如:
\end{tcolorbox}

\begin{table}[!htbp]
	\centering
	\vspace{-2ex}
	\small% fontsize
	% \setlength{\tabcolsep}{4pt}% column separation
	% \renewcommand{\arraystretch}{1.2}%row space 
	\begin{tabular}{cp{25em}p{10.5em}}
		\toprule
		\textbf{序号} & \multicolumn{1}{c}{\textbf{进度}} & \multicolumn{1}{c}{\textbf{起止时间}} \\
		\midrule
		1           & 任务下达                            & 2024.11.29 $\sim$ 2024.12.06      \\
		2           & 完成开题报告,文献综述等文档撰写工作              & 2024.12.06 $\sim$ 2024.12.27      \\
		3           & 完成开题答辩                          & 2024.12.27 $\sim$ 2025.01.02      \\
		4           & 初步完成整体设计与开发任务                   & 2025.01.02 $\sim$ 2025.04.01      \\
		5           & 完成中期检查                          & 2025.04.01 $\sim$ 2025.04.15      \\
		6           & 完成项目开发并撰写毕业论文                   & 2025.04.15 $\sim$ 2025.05.01      \\
		7           & 准备毕业设计答辩工作                      & 2025.05.01 $\sim$ 2025.05.17      \\
		8           & 完成答辩并上交材料                       & 2025.05.17 $\sim$ 2025.05.20      \\
		\bottomrule
	\end{tabular}
	\vspace{-2ex}
\end{table}

\section{主要参考文献}

\begin{tcolorbox}
	因官方并无明确要求参考文献数量,为保险起见,建议按毕业论文要求处理,即总量至
	少10篇(包括至少2篇外文文献)。
\end{tcolorbox}

% Should be empty here, since the bibliograpy will be generated automatically
% based on the citations in the main body.
