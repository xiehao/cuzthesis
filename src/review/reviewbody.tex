\begin{tcolorbox}
	\emph{郑重声明:本文档仅给出个人建议,仅供参考。另外,在实际撰写时,凡出现于
		tcolorbox环境中的文本均应连同环境本身一同注释或直接删除,以免影响最终呈
		现效果。}
\end{tcolorbox}

\begin{tcolorbox}
	\begin{itemize}
		\item 文献综述应详细阐述与毕业设计选题相关领域的国内外研究现状、发展趋
		      势、存在的问题等。
		\item 文献综述的质量从某种程度上可反映出作者对相关领域是否\emph{充分}了
		      解,进而是否有能力开始着手毕业设计工作。
		\item 文献综述应尽量做到\emph{知己知彼}、\emph{有的放矢},避免坐井观天、
		      闭门造车。
	\end{itemize}
\end{tcolorbox}

\section{国内外研究现状}

\begin{tcolorbox}
	此部分可从总体概括介绍国内外\emph{从过去到现在}相关领域的研究背景、历史发展
	与研究现状。建议可按国内外分别阐述:
\end{tcolorbox}

\subsection{国内研究现状}

\begin{tcolorbox}
	总体概括国内研究现状。
\end{tcolorbox}

\subsection{国外研究现状}

\begin{tcolorbox}
	总体概况国外研究现状。
\end{tcolorbox}

\section{研究成果}

\begin{tcolorbox}
	此部分可重点展开阐述从过去到现在的一些\emph{代表性}工作成果。在阐述时可按某
	种方式对成果适当进行\emph{归类}、\emph{分析}、\emph{比较}。建议可为每一类单
	独做小节标题\verb|\subsection{...}|,在小节中按时间顺序详细阐述该类中的重要
	工作成果,分析每类工作的优缺点、适用性等,最后再进行类间综合比较,并联系自己
	的工作得出初步结论。
\end{tcolorbox}

\section{进展情况和发展趋势}

\begin{tcolorbox}
	此部分可着重介绍\emph{从现在到将来}的技术发展趋势。可从现有成果出发,合理预
	测未来相关技术、平台等的发展趋势。也可借此机会说明自己工作与发展趋势的关系。
\end{tcolorbox}

\section{存在的问题}

\begin{tcolorbox}
	此部分可列出现有成果中存在的问题。只有有问题才有改进空间,才更能凸显自己工作
	的必要性。建议只列出\emph{少量}($3\sim5$个为宜)、且自己工作有可能解决的问
	题,同时引出自己的工作,即选题所要解决的问题与达到的基本目标。
\end{tcolorbox}

\section{主要参考文献}

\begin{tcolorbox}
	此部分无需加入任何内容,模板会自动根据正文中的引用生成参考文献,故\emph{要确
		保在正文中适当位置、以适当方式引用适当数目的适当文献},具体解释如下:
	\begin{itemize}
		\item 引用\emph{位置}应紧跟在被引内容后,但尽量不要出现在标点符号后,除
		      非整句或整段引用(不提倡);
		\item 引用\emph{方式}可采用作者上标型(如:\citet{chen1980zhongguo}提出
		      ……)、直接上标型(如:采用了……方法\citep{chen2005zhulu})或多个参
		      考文献同时引用(如:这些方法
		      \citep{Bohan1928,chu2004tushu,Dubrovin1906,hls2012jinji}……),但应
		      注意避免同时引用过多(此处为反例
		      \citep{yuan2012lana,yuan2012lanb,yuan2012lanc,wikibook2014latex,walls2013drought,stamerjohanns2009mathml,niu2013zonghe,lamport1986document});
		\item 引用\emph{数目}应满足要求,即不低于10篇(其中不低于2篇外文文献);
		\item 被引\emph{文献}应尽量选择优质且最新成果(以近3年以内为宜,若确为经
		      典可适当放宽年限)。
	\end{itemize}
\end{tcolorbox}

% Should be empty here, since the bibliograpy will be generated automatically
% based on the citations in the main body.
