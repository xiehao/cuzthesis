%---------------------------------------------------------------------------%
%-                                                                         -%
%-                       论文基本信息初始化                                -%
%-                                                                         -%
%---------------------------------------------------------------------------%
% 本文件包含论文的基本信息,这些信息将在论文的多个部分使用
% 请根据您的实际情况修改以下内容
%---------------------------------------------------------------------------%

%-> 保密级别设置(如无特殊要求,请保持为空)
% 可选值:空(无保密要求)、"秘密"、"机密"、"绝密"
\confidential{}

%-> 学校标志设置
% 第一个参数:标志宽度(相对于页面宽度的比例,推荐值:0.8)
% 第二个参数:标志文件名(不含扩展名,默认在 figures 目录下查找)
\schoollogo{0.8}{cuz_logo}

%-> 论文标题设置
% 中文标题应简明扼要,一般不超过20个汉字
% 例如:基于深度学习的视频内容分析与推荐系统研究
\title{浙江传媒学院毕业论文\LaTeX{}模板}

%-> 论文英文标题设置
% 英文标题应与中文标题内容一致,首字母大写
% 例如:Research on Video Content Analysis and Recommendation System Based on Deep Learning
\englishtitle{A \LaTeX{} Thesis Template for Communication University of Zhejiang (CUZ)}

%-> 作者信息设置
% 作者姓名应与学籍信息一致
% 例如:张三
\author{张三}

%-> 学号设置
% 请填写您的完整学号
% 例如:20230123456
\authorid{20230123456}

%-> 指导教师信息设置
% 指导教师姓名应与教师证件信息一致
% 例如:李四
\advisor{李四}

%-> 指导教师职称设置
% 常见职称:教授、副教授、讲师、助教
% 例如:教授
\advisortitle{教授}

%-> 合作/企业教师信息设置(如无可保留,使用nocoadvisor选项可在封面隐藏此项)
% 例如:王五
\coadvisor{王五}

%-> 合作/企业教师职称设置
% 常见职称:教授级高工、高级工程师、工程师等
% 例如:高级工程师
\coadvisortitle{高级工程师}

%-> 学位类别设置
% 必须是以下三者之一:学士、硕士、博士
\degree{学士}

%-> 专业名称设置
% 请填写完整的专业名称,应与学籍信息一致
% 例如:数字媒体技术
\major{数字媒体技术}

%-> 班级名称设置
% 请填写完整的班级名称,应与学籍信息一致
% 例如:数字媒体技术2020-1班
\class{数字媒体技术2020-1班}

%-> 学院名称设置
% 请填写完整的学院名称
% 例如:计算机科学与技术学院
\institute{计算机科学与技术学院}

%-> 毕业年份设置
% 请填写您的预期毕业年份
% 例如:2024
\graduateyear{2024}

%-> 开题报告日期设置(用于开题报告文档)
% 格式:YYYY年MM月DD日
% 例如:2023年10月15日
\openingdate{2023年10月15日}

%-> 文献综述日期设置(用于文献综述文档)
% 格式:YYYY年MM月DD日
% 例如:2023年12月20日
\reviewdate{2023年12月20日}
%---------------------------------------------------------------------------%
