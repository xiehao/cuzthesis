% !TEX program = xelatex
% !TEX options = -synctex=1 -interaction=nonstopmode -shell-escape -file-line-error "%DOC%"
% !BIB program = bibtex
%---------------------------------------------------------------------------%
%-                                                                         -%
%-                     浙江传媒学院毕业论文模板                           -%
%-                                                                         -%
%---------------------------------------------------------------------------%
%- Copyright (C) Hao XIE <oaheix@gmail.com>
%- This is free software: you can redistribute it and/or modify it
%- under the terms of the GNU General Public License as published by
%- the Free Software Foundation, either version 3 of the License, or
%- (at your option) any later version.
%---------------------------------------------------------------------------%
%->> 文档类声明
%---------------------------------------------------------------------------%
% 本文档使用 cuzthesis 文档类,专为浙江传媒学院毕业论文设计
% 以下是可用的选项及其说明:
%
% 【使用说明】
% 1. 首次使用时,请先阅读本文件中的注释,了解各选项的功能
% 2. 根据需要修改文档类选项和宏包选项
% 3. 在 src/common/initialization.tex 中填写论文基本信息
% 4. 在 src/thesis/frontmatter/abstracts.tex 中编写中英文摘要
% 5. 在 src/thesis/mainmatter/ 目录下编写各章节内容
% 6. 使用 ./scripts/run.sh -xa cuzthesis.tex 命令编译论文
%
\documentclass[singlesided,cuzhdr]{styles/cuzthesis}%

%->> 文档类选项说明
%---------------------------------------------------------------------------%
% 页面布局选项(三选一):
% - singlesided:单面打印模式,页边距对称
% - doublesided:双面打印模式,页边距交替变化
% - printcopy:印刷出版模式,为装订留出额外空间

% 页眉页脚选项:
% - cuzhdr:启用浙传风格的页眉页脚

% 版本信息选项:
% - draftversion:显示草稿版本信息

% 字体设置选项:
% - fontset=<fandol|founder|mac|ubuntu|windows|...>:指定字体集,替代自动检测
%   常用值:
%   - windows:使用 Windows 系统字体(宋体、黑体等)
%   - mac:使用 macOS 系统字体
%   - ubuntu:使用 Ubuntu 系统字体
%   - fandol:使用开源的 Fandol 字体(TeX Live 自带)

% 国际化选项:
% - scheme=plain:国际学生论文写作模式,使用英文界面



% 其他选项:
% - 支持 ctex book 类的标准选项:draft、paper size、font size 等
%---------------------------------------------------------------------------%
%->> 文档设置
%---------------------------------------------------------------------------%
% artratex 宏包提供了多种文档设置选项,用于控制参考文献、图表、数学等功能
% 常用配置组合:
% - 基础配置:\usepackage[super]{styles/artratex}
% - 标准配置:\usepackage[super,list,table]{styles/artratex}
% - 完整配置:\usepackage[super,list,table,tikz,math,color]{styles/artratex}
% - 作者年份引用:\usepackage[authoryear,list,table]{styles/artratex}
\usepackage[super,list,table,tikz]{styles/artratex}% 文档设置

%->> artratex 宏包选项说明
%---------------------------------------------------------------------------%
% 参考文献处理器选项(二选一):
% - bibtex:使用 BibTeX 处理参考文献(默认)
% - biber:使用 Biber 处理参考文献(支持更多功能)

% 引用样式选项(四选一):
% - numbers:数字引用样式
%   - 文内引用:Jones [1]
%   - 括号引用:[1]
% - super:上标数字引用样式(默认)
%   - 文内引用:Jones^[1]
%   - 括号引用:^[1]
% - authoryear:作者-年份引用样式
%   - 文内引用:Jones (1995)
%   - 括号引用:(Jones, 1995)
% - alpha:字母数字混合引用样式
%   - 文内引用:不可用
%   - 括号引用:[Jon95]

% 页面布局选项:
% - geometry:通过 geometry 宏包重新配置页面布局

% 环境支持选项:
% - lscape:提供横向页面布局环境
% - myhdr:通过 fancyhdr 宏包启用自定义页眉页脚
% - color:通过 xcolor 宏包提供颜色支持
% - background:启用页面背景
% - tikz:通过 tikz 宏包提供复杂图表支持
% - table:通过 ctable 宏包提供复杂表格支持
% - list:提供增强的列表环境,用于算法和代码
% - math:启用额外的数学宏包

% artracom 宏包提供了用户自定义命令,如数学符号、向量表示等
\usepackage{styles/artracom}% 用户自定义命令
%---------------------------------------------------------------------------%
%->> 文档内容
%---------------------------------------------------------------------------%
\begin{document}

%->> 初始化信息
%---------------------------------------------------------------------------%
% 导入初始化文件,包含论文的基本信息(标题、作者、学院等)
% 在 src/common/initialization.tex 中修改以下信息:
% - 论文标题(中英文)
% - 作者姓名、学号
% - 指导教师姓名、职称
% - 学院、专业、班级
% - 毕业年份
%---------------------------------------------------------------------------%
%-                                                                         -%
%-                       论文基本信息初始化                                -%
%-                                                                         -%
%---------------------------------------------------------------------------%
% 本文件包含论文的基本信息,这些信息将在论文的多个部分使用
% 请根据您的实际情况修改以下内容
%---------------------------------------------------------------------------%

%-> 保密级别设置(如无特殊要求,请保持为空)
% 可选值:空(无保密要求)、"秘密"、"机密"、"绝密"
\confidential{}

%-> 学校标志设置
% 第一个参数:标志宽度(相对于页面宽度的比例,推荐值:0.8)
% 第二个参数:标志文件名(不含扩展名,默认在 figures 目录下查找)
\schoollogo{0.8}{cuz_logo}

%-> 论文标题设置
% 中文标题应简明扼要,一般不超过20个汉字
% 例如:基于深度学习的视频内容分析与推荐系统研究
\title{浙江传媒学院毕业论文\LaTeX{}模板}

%-> 论文英文标题设置
% 英文标题应与中文标题内容一致,首字母大写
% 例如:Research on Video Content Analysis and Recommendation System Based on Deep Learning
\englishtitle{A \LaTeX{} Thesis Template for Communication University of Zhejiang (CUZ)}

%-> 作者信息设置
% 作者姓名应与学籍信息一致
% 例如:张三
\author{张三}

%-> 学号设置
% 请填写您的完整学号
% 例如:20230123456
\authorid{20230123456}

%-> 指导教师信息设置
% 指导教师姓名应与教师证件信息一致
% 例如:李四
\advisor{李四}

%-> 指导教师职称设置
% 常见职称:教授、副教授、讲师、助教
% 例如:教授
\advisortitle{教授}

%-> 合作/企业教师信息设置(如无可保留,使用nocoadvisor选项可在封面隐藏此项)
% 例如:王五
\coadvisor{王五}

%-> 合作/企业教师职称设置
% 常见职称:教授级高工、高级工程师、工程师等
% 例如:高级工程师
\coadvisortitle{高级工程师}

%-> 学位类别设置
% 必须是以下三者之一:学士、硕士、博士
\degree{学士}

%-> 专业名称设置
% 请填写完整的专业名称,应与学籍信息一致
% 例如:数字媒体技术
\major{数字媒体技术}

%-> 班级名称设置
% 请填写完整的班级名称,应与学籍信息一致
% 例如:数字媒体技术2020-1班
\class{数字媒体技术2020-1班}

%-> 学院名称设置
% 请填写完整的学院名称
% 例如:计算机科学与技术学院
\institute{计算机科学与技术学院}

%-> 毕业年份设置
% 请填写您的预期毕业年份
% 例如:2024
\graduateyear{2024}

%-> 开题报告日期设置(用于开题报告文档)
% 格式:YYYY年MM月DD日
% 例如:2023年10月15日
\openingdate{2023年10月15日}

%-> 文献综述日期设置(用于文献综述文档)
% 格式:YYYY年MM月DD日
% 例如:2023年12月20日
\reviewdate{2023年12月20日}
%---------------------------------------------------------------------------%


%->> 前置部分:封面、声明、摘要、目录
%---------------------------------------------------------------------------%
% 设置页面布局为普通格式(无页眉页脚)
\plainmatter

% 生成论文封面
% 封面内容根据 initialization.tex 中的信息自动生成
% 如需修改封面样式,请修改 cuzthesis.cls 文件中的 \maketitle 命令
\maketitle

% 生成作者声明页
% 如需修改声明内容,请修改 cuzthesis.cls 文件中的 \makedeclaration 命令
\makedeclaration

% 更改页面布局为无页脚格式(有页眉无页脚)
\nofootermatter

% 导入中英文摘要
% 在 src/thesis/frontmatter/abstracts.tex 中编写中英文摘要和关键词
% 摘要应简明扼要地概述论文的主要内容,包括研究目的、方法、结果和结论
% 关键词应能反映论文的主题和特点,一般为 3-8 个
\begin{chineseabstract}
    {浙江传媒学院,毕业论文,\hologo{LaTeX}模板,全文的浓缩}本文是浙江传媒学院本
    科毕业论文模板cuzthesis的使用说明文档。主要内容为介绍\hologo{LaTeX}文档类
    cuzthesis的用法,以及撰写毕业论文的一些拙见。

    \begin{leftbar}
        \noindent\textbf{建议:}摘要是全文的浓缩,要涵盖文章的主要内容,一般应保
        证字数在300以上,须包括如下三点:做了什么、如何做的、做得怎样。具体而
        言:\textbf{第一段},可以用一句话概述课题背景;接着再用一两句话表明做的
        东西。\textbf{第二段},可以用三五句话描述如何做的,包括做这个东西分几步
        (或几个模块),每步(模块)的任务,以及相邻步(或相关模块)之间的关系。
        \textbf{最后一段},用两三句话概括作品的亮点,以及效果怎样,包括可能的市
        场反响、用户体验等等。
        
        \noindent{}需要注意的是,既然是全文的浓缩,就不要出现类似“本项目将要做
        到”或者“详情见正文”等此类语句,因其并非正文中应出现的说法(既然已经是正
        文的浓缩,再说详情见正文就产生了矛盾)。此外,语言上应注意言简意赅,语气
        上应平实无感。
    \end{leftbar}
\end{chineseabstract}
\begin{englishabstract}
    {Communication University of Zhejiang (CUZ), thesis, \LaTeX{} Template,
    abstraction of the whole thesis} This documentation is a helper for the
    \LaTeX{} class \texttt{cuzthesis}, which is a thesis template for the
    Communication University of Zhejiang. The main content is about how to use
    the \texttt{cuzthesis}, as well as some advices of writing thesis.

    \begin{leftbar}
        \noindent\textbf{Suggestions:} Abstract is the abstraction of the whole
        thesis, which should cover the main aspects of the thesis, and should
        contain no less than 300 words, involving 3 aspects: what is done, how
        to do it, and what about the result. Specifically, \textbf{1st
        paragraph}, introduce the background by 1 sentence, and what is done by
        1 or 2 sentence(s). \textbf{2nd paragraph}, describe how to do it by 3
        to 5 sentences, including how many steps (modules) are used, what each
        step (module) does, and the relationship between adjacent steps (related
        modules). \textbf{Last paragraph}, extract the most important features
        of the work and its effect by 2 or 3 sentences, including possible
        responses of markets, user experiences, etc..
    	
    	\noindent Note that, since it is the abstraction of the whole thesis,
    	expressions such as ``this project will'' or ``readers are refered to
    	the body of thesis for details'' should be avoided, for this saying
    	should not appear in the body of thesis (now that this is the
    	abstraction, a refering to the body of thesis generates a conflict).
    	Besides, the language should be neat, and the tone should be plain and
    	without emotions.
    \end{leftbar}
\end{englishabstract}

% 生成目录
% 目录自动根据各级标题生成,无需手动编辑
% 如需添加图表目录,可在此处添加 \listoffigures 和 \listoftables 命令
\tableofcontents
% \listoffigures  % 图目录(可选)
% \listoftables   % 表目录(可选)
% \listoflistings % 代码目录(可选)

%->> 主体部分:论文正文
%---------------------------------------------------------------------------%
% 更改页面布局为主体格式(有页眉页脚)
\mainmatter

% 导入论文主体内容(包含所有章节)
% mainbody.tex 文件通过 \input 命令引用各个章节文件
% 如需添加或删除章节,请修改 src/thesis/mainmatter/mainbody.tex 文件
% 各章节文件位于 src/thesis/mainmatter/ 目录下
%---------------------------------------------------------------------------%
%->>>>>>>     This part should be deleted in the final thesis.      <<<<<<<-%
%---------------------------------------------------------------------------%
\vspace*{\fill}
\begin{leftbar}
	\noindent{}本文以\href{https://github.com/mohuangrui/ucasthesis}{ucasthesis}
	使用指南为基础,保留了行文结构与多数内容,并按浙传毕业论文要求做了相应修改,
	加入了一些意见与建议。在此向原作者莫晃锐及相关人士表示忠心感谢!
\end{leftbar}
\vspace*{\fill}
%---------------------------------------------------------------------------%
%->>>>>>>     This part should be deleted in the final thesis.      <<<<<<<-%
%---------------------------------------------------------------------------%

%---------------------------------------------------------------------------%
%->> 论文主体部分
%---------------------------------------------------------------------------%
% 论文主体部分通常包括以下几个章节:
% 1. 绪论:介绍研究背景、意义、目的、内容和方法
% 2. 文献综述:回顾和评述相关研究成果
% 3. 模板设计与实现:详细介绍模板的设计思路和实现方法
% 4. 使用指南与示例:提供模板的使用说明和示例
% 5. 结论:总结研究成果和展望

% \chapter{绪论}\label{chap:introduction}
\begin{cuzchapter}{绪论}{chap:introduction}

	%---------------------------------------------------------------------------%
	%->> 绪论写作指南
	%---------------------------------------------------------------------------%
	% 绪论是论文的第一章,主要介绍研究的背景、意义、目的、内容和方法等
	% 一个完整的绪论应包含以下几个部分:
	% 1. 研究背景:介绍研究领域的发展历程和现状,指出存在的问题
	% 2. 研究意义:阐述开展本研究的理论意义和实践价值
	% 3. 研究目的:明确说明本研究要解决的具体问题
	% 4. 研究内容:概述论文的主要研究内容和章节安排
	% 5. 研究方法:简要介绍本研究采用的主要研究方法

	\section{研究背景}\label{sec:background}

	随着信息技术的快速发展和学术交流的日益国际化,高质量的学术论文写作和排版变得越来越重要。在当前的学术环境中,一篇内容充实、格式规范的学术论文不仅能够更好地传达研究成果,还能提高研究工作的专业性和可信度。然而,许多学生在论文写作过程中常常面临两大挑战:一是如何组织和表达研究内容,二是如何处理复杂的排版格式要求。

	传统的文字处理软件(如Microsoft Word)虽然使用广泛,但在处理大型学术文档时存在诸多局限性,如格式不一致、公式排版困难、参考文献管理复杂等问题。相比之下,\LaTeX{}作为一种专业的排版系统,以其卓越的数学公式处理能力、自动化的参考文献管理、精确的版面控制和内容与格式分离的特性,在学术界获得了广泛应用。然而,\LaTeX{}的学习曲线较陡峭,许多初学者往往因其复杂性而望而却步。

	考虑到许多同学可能缺乏\LaTeX{}使用经验,cuzthesis在参考了ucasthesis模板的基础上,将\LaTeX{}的复杂性高度封装,开放出简单的接口,以便轻易使用。同时,对用\LaTeX{}撰写论文的一些主要难题,如制图、制表、文献索引等,进行了详细说明,并提供了相应的代码样本,理解了上述问题后,对于初学者而言,使用此模板撰写学位论文将不存在实质性的困难。

	此浙传毕业论文模板cuzthesis基于国科大莫晃锐制作的ucasthesis模板发展而来。当前cuzthesis模板满足最新的浙江传媒学院本科毕业论文撰写要求和封面设定,兼顾主流操作系统:Windows,Linux,macOS 和主流\LaTeX{}编译引擎:\hologo{pdfLaTeX}、 \hologo{XeLaTeX}、\hologo{LuaLaTeX},详细支持情况见表\ref{tab:support-status}。支持中文书签、中文渲染、中文粗体显示、拷贝PDF中的文本到其他文本编辑器等特性。此外,对模板的文档结构进行了精心设计,撰写了编译脚本提高模板的易用性和使用效率。

	\section{研究意义}\label{sec:significance}

	开发和完善浙江传媒学院毕业论文\LaTeX{}模板具有重要的理论和实践意义:

	\subsection{理论意义}

	\begin{enumerate}
		\item \textbf{促进学术规范化}:统一的论文模板有助于规范学术写作格式,提高学术成果的专业性和可读性,促进学术交流的标准化。

		\item \textbf{推动技术教育创新}:引导学生使用\LaTeX{}等专业工具进行学术写作,有助于培养学生的技术素养和创新能力,拓展其专业技能范围。

		\item \textbf{支持内容与形式分离}:\LaTeX{}的"内容与形式分离"理念有助于学生更加专注于研究内容本身,而非繁琐的排版细节,从而提高研究质量。
	\end{enumerate}

	\subsection{实践意义}

	\begin{enumerate}
		\item \textbf{提高论文质量}:规范化的模板和专业的排版工具可以显著提高毕业论文的整体质量和美观度,减少格式错误。

		\item \textbf{节省时间和精力}:自动化的参考文献管理、章节编号、目录生成等功能可以大大减少学生在排版上花费的时间和精力。

		\item \textbf{增强学术竞争力}:掌握\LaTeX{}等专业工具可以增强学生的学术竞争力,为其未来的学术研究或职业发展奠定基础。

		\item \textbf{促进学校学术形象}:统一规范的高质量论文格式有助于提升学校的学术形象和声誉。
	\end{enumerate}
	\begin{table}[htbp]
		\caption[编译引擎跨平台情况]{各平台下编译引擎支持情况(\checkmark:支持或部分支持;$\times$:不支持)}
		\label{tab:support-status}
		\centering
		\small% fontsize
		% \setlength{\tabcolsep}{4pt}% column separation
		% \renewcommand{\arraystretch}{1.2}%row space
		\begin{tabular}{cccc}
			\toprule
			                     & \hologo{pdfLaTeX}                          & \hologo{XeLaTeX}                     & \hologo{LuaLaTeX} \\
			\midrule
			Linux                & $\times$                                   & \checkmark\footnote{暂不完全支持,粗楷体加由粗宋体代
			替,仿宋加粗无效;但不影响本模板使用。} & \checkmark\footnote{暂不完全支
			持,粗楷体加由粗宋体代替,仿宋加粗无效;但不影响本模板使用。}                                                                                              \\
			macOS                & $\times$                                   & \checkmark\footnote{暂不完全支持,仿宋加粗无效;但不
			影响本模板使用。}            & $\times$                                                                                              \\
			Windows              & \checkmark\footnote{暂不完全支持,粗宋体加由黑体代替。}     &
			\checkmark\footnote{暂不完全支持,粗楷体由粗宋体代替;但不影响本模板使
			用。}                  & \checkmark\footnote{暂不完全支持,所有中文字体均无法加粗,且编译
			时间较\hologo{XeLaTeX}慢一些。}                                                                                                     \\
			\bottomrule
		\end{tabular}
	\end{table}

	\section{研究目的}\label{sec:purpose}

	本研究旨在开发一个全面、易用且符合学术标准的浙江传媒学院毕业论文\LaTeX{}模板,具体目的包括:

	\begin{enumerate}
		\item 创建一个符合浙江传媒学院最新毕业论文格式要求的\LaTeX{}模板,确保学生提交的论文格式规范统一。

		\item 简化\LaTeX{}的使用流程,降低学习门槛,使初学者能够快速掌握模板的使用方法。

		\item 提供全面的论文写作指导,帮助学生理解学术论文的结构和内容要求,提高论文质量。

		\item 构建一个可扩展的模板框架,便于未来根据学校要求的变化进行更新和维护。

		\item 推广\LaTeX{}在学术写作中的应用,提高学生的学术写作能力和技术素养。
	\end{enumerate}

	\section{研究内容}\label{sec:content}

	本研究的主要内容包括以下几个方面:

	\subsection{模板设计与开发}

	cuzthesis的目标在于简化毕业论文的撰写,利用\LaTeX{}格式与内容分离的特征,模板将格式设计好后,作者可只需关注论文内容。同时,cuzthesis有着整洁一致的代码结构和扼要的注解,对文档的仔细阅读可为初学者提供一个学习\LaTeX{}的窗口。此外,模板的架构十分注重通用性,事实上,与ucasthesis一样,cuzthesis不仅是浙传毕业论文模板,同时,通过少量修改即可成为使用\LaTeX{}撰写中英文文章或书籍的通用模板,并为使用者的个性化设定提供了接口。

	\subsection{论文写作指导}

	本研究不仅提供了\LaTeX{}模板,还包含了全面的论文写作指导,涵盖以下内容:

	\begin{itemize}
		\item \textbf{论文结构指导}:详细介绍学术论文的标准结构,包括绪论、文献综述、研究方法、结果分析、讨论和结论等部分的写作要点。

		\item \textbf{学术写作规范}:提供学术写作的基本规范,包括语言表达、引用格式、图表制作等方面的指导。

		\item \textbf{常见问题解答}:针对论文写作和\LaTeX{}使用过程中的常见问题提供解答和建议。
	\end{itemize}

	\subsection{使用示例与案例}

	为了帮助用户更好地理解和使用模板,本研究提供了丰富的使用示例和案例,包括:

	\begin{itemize}
		\item 各类型内容(文本、公式、图表、代码等)的排版示例
		\item 不同学科论文的格式案例
		\item 常见排版问题的解决方案
	\end{itemize}

	\section{研究方法}\label{sec:method}

	本研究采用以下方法进行模板的设计、开发和完善:

	\subsection{文献研究法}

	通过查阅国内外相关文献,了解\LaTeX{}模板设计的最新进展和最佳实践,为模板开发提供理论基础。同时,研究浙江传媒学院的论文格式要求和学术规范,确保模板符合学校标准。

	\subsection{比较分析法}

	对比分析国内外知名高校的\LaTeX{}论文模板,如清华大学、北京大学、中国科学院大学等机构的模板,吸取其优点,避免其不足,为模板设计提供参考。

	\subsection{实验开发法}

	采用迭代开发的方式,不断测试和完善模板功能,确保模板的稳定性和易用性。在开发过程中,注重用户体验,简化操作流程,降低学习门槛。

	\subsection{用户反馈法}

	收集和分析用户反馈,了解用户在使用过程中遇到的问题和需求,及时进行修复和改进,不断提高模板的质量和用户满意度。

	\section{系统要求}\label{sec:system}

	\href{https://github.com/xiehao/CUZThesis}{cuzthesis}宏包可以在目前主流的
	\href{https://en.wikibooks.org/wiki/LaTeX/Introduction}{\LaTeX{}}编译系统中
	使用,例如C\TeX{}套装(请勿混淆C\TeX{}套装与ctex宏包。C\TeX{}套装是集成了许
	多\LaTeX{}组件的\LaTeX{}编译系统,因已停止维护,\textbf{不再建议使用}。
	\href{https://ctan.org/pkg/ctex?lang=en}{ctex} 宏包如同cuzthesis,是\LaTeX{}
	命令集,其维护状态活跃,并被主流的\LaTeX{}编译系统默认集成,是几乎所有
	\LaTeX{}中文文档的核心架构。)、MiK\TeX{}(维护较不稳定,\textbf{不太推荐使
	用})、\TeX{}Live。而文本编辑器方面则包括:Visual Studio Code(简称VS
	Code)、\hologo{TeX}studio、 Emacs、Vim等。推荐的
	\href{https://en.wikibooks.org/wiki/LaTeX/Installation}{\LaTeX{}编译系统}和
	\href{https://en.wikibooks.org/wiki/LaTeX/Installation}{\LaTeX{}文本编辑器}
	如表\ref{tab:recomendations}所示。
	\begin{table}[htbp]
		\caption[推荐的编译系统与编辑器]{不同平台下推荐的编译系统与编辑器}
		\label{tab:recomendations}
		\centering
		\small% fontsize
		%\setlength{\tabcolsep}{4pt}% column separation
		%\renewcommand{\arraystretch}{1.5}% row space
		\begin{tabular}{ccc}
			\toprule
			%\multicolumn{num_of_cols_to_merge}{alignment}{contents} \\
			%\cline{i-j}% partial hline from column i to column j
			操作系统    & \LaTeX{}编译系统                                            & \LaTeX{}文本编辑器 \\
			\midrule
			Linux   &
			\href{https://www.tug.org/texlive/acquire-netinstall.html}{\TeX{}Live
			Full}   & \href{https://code.visualstudio.com/download}{VS Code}、
			\href{https://www.texstudio.org/}{\hologo{TeX}studio}、Emacs、Vim                   \\
			MacOS   & \href{https://www.tug.org/mactex/}{Mac\TeX{} Full}      &
			\href{https://code.visualstudio.com/download}{VS Code}、
			\href{https://www.texstudio.org/}{\hologo{TeX}studio}、Emacs                       \\
			Windows &
			\href{https://www.tug.org/texlive/acquire-netinstall.html}{\TeX{}Live
			Full}   & \href{https://code.visualstudio.com/download}{VS Code}、
			\href{https://www.texstudio.org/}{\hologo{TeX}studio}                             \\
			\bottomrule
		\end{tabular}
	\end{table}

	\LaTeX{}编译系统,如\TeX{}Live(Mac\TeX{}为针对macOS的\TeX{}Live),用于提供
	编译环境;\LaTeX{}文本编辑器 (如VS Code) 用于编辑\TeX{}源文件。请从各软件官
	网下载安装程序,勿使用不明程序源。\textbf{\LaTeX{}编译系统和\LaTeX{}编辑器分
	别安装成功后,即完成了\LaTeX{}的系统配置},无需其他手动干预和配置。若系统原
	带有旧版的\LaTeX{}编译系统并想安装新版,请\textbf{先卸载干净旧版再安装新
	版}。

	\section{论文结构}\label{sec:structure}

	本论文共分为五章,各章内容概述如下:

	\textbf{第一章 绪论}:介绍研究背景、研究意义、研究目的、研究内容和研究方法,阐述开发浙江传媒学院毕业论文\LaTeX{}模板的必要性和重要性,并概述系统要求和使用环境。

	\textbf{第二章 文献综述}:回顾和评述国内外相关研究成果,包括\LaTeX{}在学术写作中的应用、国内外高校论文模板的发展现状以及学术论文写作规范等方面的研究,为模板开发提供理论基础。

	\textbf{第三章 模板设计与实现}:详细介绍模板的设计思路、架构和实现方法,包括文档类的设计、页面布局、字体设置、章节样式、图表格式、参考文献管理等方面的技术实现。

	\textbf{第四章 使用指南与示例}:提供全面的模板使用指南,包括环境配置、基本用法、高级功能和常见问题解决方案,并通过丰富的示例展示模板的各种功能和应用场景。

	\textbf{第五章 总结与展望}:总结模板开发的主要成果和特点,分析存在的不足,并对未来的改进方向和发展前景进行展望。

	此外,论文还包括参考文献、致谢和附录等部分,提供补充信息和资源。



	\section{问题反馈}\label{sec:callback}

	关于\LaTeX{}的知识性问题,请查阅
	\href{https://github.com/mohuangrui/ucasthesis/wiki}{ucasthesis和\LaTeX{}知
	识小站} 和 \href{https://en.wikibooks.org/wiki/LaTeX}{\LaTeX{} Wikibook}。

	关于模板编译和样式设计的问题,请\textbf{先仔细阅读此说明文档,特别是“常见问
		题” (章节~\ref{sec:qa})}。若问题仍无法得到解决,请\textbf{先将问题理解
		清楚并描述清楚,再将问题反馈}至
		\href{https://github.com/xiehao/CUZThesis/issues}{Github/cuzthesis/issues}。

	欢迎大家有效地反馈模板不足之处,一起不断改进模板。希望大家向同事积极推广
	\LaTeX{},一起更高效地做科研。

	\section{模板下载}\label{sec:download}

	\begin{center}
		\href{https://github.com/xiehao/CUZThesis}{Github/cuzthesis}:
		\url{https://github.com/xiehao/CUZThesis}。
	\end{center}

\end{cuzchapter}% 绪论
\begin{cuzchapter}{文献综述}{chap:literature}

	%---------------------------------------------------------------------------%
	%->> 文献综述写作指南
	%---------------------------------------------------------------------------%
	% 文献综述是论文的重要组成部分,主要回顾和评述与研究主题相关的已有研究成果
	% 一个完整的文献综述应包含以下几个部分:
	% 1. 研究领域概述:介绍研究领域的基本概念、发展历程和重要性
	% 2. 研究现状分析:分析国内外相关研究的主要成果、研究方法和研究趋势
	% 3. 研究评述:评价已有研究的优缺点,指出研究中存在的问题和不足
	% 4. 研究展望:提出未来研究的方向和可能的突破点

	\section{LaTeX在学术写作中的应用}\label{sec:latex-academic}
	
	\subsection{LaTeX的发展历程}
	
	\LaTeX{}作为一种专业的排版系统,自其诞生以来就与学术写作紧密相连。\citet{lamport1986document}在1986年基于Donald Knuth开发的\TeX{}系统创建了\LaTeX{},旨在简化复杂文档的排版过程。经过数十年的发展,\LaTeX{}已经成为科学、技术和数学领域学术论文写作的标准工具。
	
	\LaTeX{}的核心理念是"内容与形式分离",这一理念使得作者可以专注于内容创作,而不必过多关注排版细节。这种分离不仅提高了写作效率,还确保了文档的一致性和专业性。随着计算机技术的发展,\LaTeX{}也在不断更新和完善,如今已发展到\LaTeX2e和\LaTeX3阶段,提供了更强大的功能和更友好的用户体验。
	
	\subsection{LaTeX的优势分析}
	
	相比传统的文字处理软件,\LaTeX{}在学术写作中具有显著优势。\citet{stamerjohanns2009mathml}的研究表明,\LaTeX{}在数学公式排版、参考文献管理、长文档处理等方面表现出色,特别适合包含复杂数学公式和大量引用的学术论文。
	
	\LaTeX{}的主要优势包括:
	
	\begin{itemize}
		\item \textbf{高质量的排版效果}:\LaTeX{}使用专业的排版算法,能够生成高质量的文档,特别是在数学公式、表格和图形的排版方面表现卓越。
		
		\item \textbf{强大的参考文献管理}:通过BibTeX或BibLaTeX系统,\LaTeX{}可以轻松处理大量参考文献,并自动生成符合各种学术期刊要求的参考文献格式。
		
		\item \textbf{优秀的跨平台兼容性}:\LaTeX{}文档可以在不同操作系统上编译,生成的PDF文件在各种设备上显示一致。
		
		\item \textbf{良好的版本控制支持}:\LaTeX{}文档是纯文本文件,便于使用Git等版本控制系统进行管理,有利于团队协作和论文修订。
		
		\item \textbf{丰富的宏包生态系统}:\LaTeX{}拥有大量专业宏包,可以满足各种特殊排版需求,如化学式、音乐符号、棋谱等。
	\end{itemize}
	
	\subsection{LaTeX在不同学科中的应用}
	
	\LaTeX{}最初主要应用于数学、物理等理工科领域,但随着其功能的不断扩展,现已广泛应用于各个学科。\citet{hls2012jinji}的调查显示,在计算机科学、数学、物理学等领域,超过80\%的学术论文使用\LaTeX{}排版;在经济学、语言学等社会科学领域,\LaTeX{}的使用率也在逐年提高。
	
	在中国,\LaTeX{}的应用也日益广泛。\citet{chen1980zhongguo}早在20世纪80年代就开始推广\LaTeX{}在中文科技文献排版中的应用。近年来,随着中文\LaTeX{}支持的完善,越来越多的中国高校和研究机构开始采用\LaTeX{}作为学位论文和学术期刊的排版工具。
	
	\section{高校论文模板的发展现状}\label{sec:template-status}
	
	\subsection{国际高校论文模板概况}
	
	国际知名高校普遍重视\LaTeX{}论文模板的开发和维护。哈佛大学、麻省理工学院、斯坦福大学等顶尖高校都提供了官方的\LaTeX{}论文模板,这些模板不仅符合学校的格式要求,还提供了丰富的使用指南和示例。
	
	这些国际高校的论文模板通常具有以下特点:
	
	\begin{itemize}
		\item \textbf{标准化程度高}:模板严格遵循学校的格式规范,确保论文符合提交要求。
		
		\item \textbf{用户友好性强}:提供详细的使用文档和示例,降低学习门槛。
		
		\item \textbf{维护更新及时}:有专门的团队或志愿者负责模板的维护和更新,确保模板与最新的格式要求保持一致。
		
		\item \textbf{社区支持完善}:大多数模板都有活跃的用户社区,提供技术支持和问题解答。
	\end{itemize}
	
	\subsection{国内高校论文模板发展}
	
	近年来,国内高校的\LaTeX{}论文模板也取得了长足发展。清华大学、北京大学、中国科学院大学等知名高校都开发了各自的\LaTeX{}论文模板,并在GitHub等平台开源,供学生和研究人员使用。
	
	\citet{niu2013zonghe}对国内30所高校的\LaTeX{}论文模板进行了综合评价,结果表明,国内高校论文模板在功能完善度、用户友好性和社区活跃度等方面与国际知名高校相比仍有一定差距,但发展速度较快,部分高校的模板已经达到了较高水平。
	
	国内高校论文模板的主要特点包括:
	
	\begin{itemize}
		\item \textbf{中文支持完善}:针对中文排版的特殊需求进行了优化,如标点符号、行间距、字体等。
		
		\item \textbf{符合国家标准}:遵循GB/T 7713-2014《学位论文编写规则》等国家标准,确保论文格式规范。
		
		\item \textbf{开源共享}:大多数模板采用开源许可证发布,便于用户修改和定制。
		
		\item \textbf{社区驱动}:主要由学生和校友自发开发和维护,社区参与度高。
	\end{itemize}
	
	\section{学术论文写作规范}\label{sec:writing-standards}
	
	\subsection{国际学术写作规范}
	
	国际学术写作有着严格的规范和标准,这些规范不仅涉及论文的格式,还包括内容组织、语言表达、引用方式等多个方面。常见的国际学术写作规范包括:
	
	\begin{itemize}
		\item \textbf{APA风格}:美国心理学会(American Psychological Association)制定的写作规范,主要用于心理学、教育学、社会科学等领域。
		
		\item \textbf{MLA风格}:现代语言协会(Modern Language Association)制定的写作规范,主要用于人文学科,如文学、语言学、艺术等。
		
		\item \textbf{Chicago风格}:芝加哥大学出版社制定的写作规范,分为注释-书目体系和作者-日期体系两种,广泛应用于各个学科。
		
		\item \textbf{IEEE风格}:电气电子工程师学会(Institute of Electrical and Electronics Engineers)制定的写作规范,主要用于工程技术领域。
	\end{itemize}
	
	\subsection{中国学术写作规范}
	
	中国的学术写作规范主要基于国家标准和行业标准,如GB/T 7713-2014《学位论文编写规则》、GB/T 7714-2015《信息与文献 参考文献著录规则》等。这些标准对论文的结构、格式、引用方式等做出了明确规定。
	
	\citet{chu2004tushu}指出,中国学术写作规范与国际规范既有共同点,也有差异。共同点在于都强调学术诚信、逻辑严密、表达准确等基本原则;差异主要体现在格式要求、引用方式和语言表达等方面。
	
	\subsection{学术写作中的常见问题}
	
	学术写作中常见的问题包括:
	
	\begin{itemize}
		\item \textbf{结构不清晰}:论文结构混乱,逻辑不连贯,各部分之间缺乏有机联系。
		
		\item \textbf{文献引用不规范}:引用格式不统一,引用内容与正文不匹配,参考文献不完整等。
		
		\item \textbf{语言表达不准确}:用词不精确,句式不规范,专业术语使用不当等。
		
		\item \textbf{格式不符合要求}:页面设置、字体大小、行间距、图表编号等不符合规定。
		
		\item \textbf{学术不端行为}:抄袭、伪造数据、重复发表等违反学术道德的行为。
	\end{itemize}
	
	\section{研究评述与展望}\label{sec:review-prospect}
	
	\subsection{现有研究的不足}
	
	通过对现有文献的综述,可以发现以下几个方面的不足:
	
	\begin{itemize}
		\item \textbf{用户体验研究不足}:大多数\LaTeX{}模板的开发主要关注功能实现和格式符合度,对用户体验的研究相对较少。
		
		\item \textbf{初学者支持不完善}:现有模板对\LaTeX{}初学者的支持不够全面,缺乏系统的学习指导和错误处理机制。
		
		\item \textbf{学术写作指导缺乏}:大多数模板只提供技术使用说明,缺乏对学术写作本身的指导。
		
		\item \textbf{跨学科适应性不强}:模板设计往往偏向特定学科,对其他学科的特殊需求考虑不足。
		
		\item \textbf{维护更新机制不完善}:许多模板缺乏长期维护机制,无法及时适应学校要求的变化。
	\end{itemize}
	
	\subsection{未来研究方向}
	
	基于上述不足,未来的研究可以从以下几个方向展开:
	
	\begin{itemize}
		\item \textbf{提升用户体验}:通过用户调研和反馈,优化模板的使用流程和界面设计,提高用户满意度。
		
		\item \textbf{加强初学者支持}:开发更完善的教程和示例,设计智能错误提示系统,降低学习门槛。
		
		\item \textbf{整合学术写作指导}:将学术写作规范和技巧融入模板,提供全面的论文写作指导。
		
		\item \textbf{增强跨学科适应性}:设计模块化的模板结构,支持不同学科的特殊需求,提高模板的通用性。
		
		\item \textbf{建立长效维护机制}:构建开源社区驱动的维护模式,确保模板的持续更新和改进。
	\end{itemize}
	
	\subsection{本研究的创新点}
	
	本研究在开发浙江传媒学院毕业论文\LaTeX{}模板的过程中,将重点关注以下创新点:
	
	\begin{itemize}
		\item \textbf{用户中心设计}:以用户需求为中心,通过调研和测试,优化模板的使用体验。
		
		\item \textbf{全面的学术写作指导}:不仅提供技术使用说明,还包含详细的学术写作指导,帮助学生提高论文质量。
		
		\item \textbf{模块化架构}:采用模块化设计,便于用户根据需要进行定制和扩展。
		
		\item \textbf{智能错误处理}:设计友好的错误提示和处理机制,帮助用户快速解决问题。
		
		\item \textbf{社区协作模式}:建立开源社区,鼓励用户参与模板的改进和维护,确保模板的可持续发展。
	\end{itemize}

\end{cuzchapter}
% 文献综述
\begin{cuzchapter}{模板设计与实现}{chap:design}

	%---------------------------------------------------------------------------%
	%->> 模板设计与实现章节写作指南
	%---------------------------------------------------------------------------%
	% 本章主要介绍模板的设计思路、架构和实现方法
	% 一个完整的模板设计与实现章节应包含以下几个部分:
	% 1. 设计目标:明确模板的设计目标和原则
	% 2. 系统架构:介绍模板的整体架构和各组件的功能
	% 3. 关键技术:详细说明模板实现中使用的关键技术
	% 4. 实现细节:介绍模板的具体实现方法和过程
	% 5. 测试与评估:说明模板的测试方法和评估结果

	\section{设计目标与原则}\label{sec:design-goals}
	
	\subsection{设计目标}
	
	cuzthesis 模板的设计目标是创建一个全面、易用且符合学术标准的浙江传媒学院毕业论文 LaTeX 模板,具体目标包括:
	
	\begin{enumerate}
		\item \textbf{格式符合性}:严格遵循浙江传媒学院毕业论文的格式要求,确保生成的论文符合学校标准。
		
		\item \textbf{易用性}:降低 LaTeX 的使用门槛,使初学者能够快速上手,专注于论文内容而非排版细节。
		
		\item \textbf{可扩展性}:采用模块化设计,便于未来根据需求进行扩展和定制。
		
		\item \textbf{跨平台兼容性}:确保模板在不同操作系统和 LaTeX 编译环境下都能正常工作。
		
		\item \textbf{文档完备性}:提供详细的使用文档和示例,帮助用户解决常见问题。
	\end{enumerate}
	
	\subsection{设计原则}
	
	在模板设计过程中,我们遵循以下设计原则:
	
	\begin{enumerate}
		\item \textbf{内容与形式分离}:严格遵循 LaTeX 的"内容与形式分离"理念,使用户能够专注于内容创作。
		
		\item \textbf{模块化设计}:将模板分为多个功能模块,每个模块负责特定的功能,便于维护和扩展。
		
		\item \textbf{用户友好性}:提供简洁明了的接口和详细的注释,降低学习成本。
		
		\item \textbf{错误容忍性}:设计合理的错误处理机制,提供友好的错误提示,帮助用户快速定位和解决问题。
		
		\item \textbf{可维护性}:采用清晰的代码结构和命名规范,便于后续维护和更新。
	\end{enumerate}
	
	\section{系统架构}\label{sec:system-architecture}
	
	cuzthesis 模板采用分层架构设计,将不同功能模块分离,以提高代码的可维护性和可扩展性。整体架构如图 \ref{fig:architecture} 所示。
	
	\begin{figure}[htbp]
		\centering
		\begin{tikzpicture}[
			block/.style={rectangle, draw, fill=blue!20, 
				text width=5cm, text centered, rounded corners, minimum height=1cm},
			line/.style={draw, -latex'},
			cloud/.style={draw, ellipse, fill=red!20, 
				minimum height=1cm, minimum width=2cm}
			]
			
			% 顶层:用户接口
			\node[block] (user) at (0,0) {用户接口层\\(cuzthesis.tex)};
			
			% 第二层:文档类和配置
			\node[block] (class) at (0,-2) {文档类层\\(cuzthesis.cls, cuzthesis.cfg)};
			
			% 第三层:功能模块
			\node[block] (function) at (0,-4) {功能模块层\\(artratex.sty, artracom.sty)};
			
			% 第四层:内容模块
			\node[block] (content) at (0,-6) {内容模块层\\(各章节文件、参考文献等)};
			
			% 连接
			\path[line] (user) -- (class);
			\path[line] (class) -- (function);
			\path[line] (function) -- (content);
			
		\end{tikzpicture}
		\caption{cuzthesis 模板架构图}
		\label{fig:architecture}
	\end{figure}
	
	\subsection{用户接口层}
	
	用户接口层是用户与模板交互的主要界面,主要包括:
	
	\begin{itemize}
		\item \textbf{cuzthesis.tex}:主文档文件,用于设置文档类选项、加载宏包和组织文档结构。
		
		\item \textbf{编译脚本}:提供便捷的编译命令,如 run.bat(Windows)和 run.sh(Linux/macOS)。
	\end{itemize}
	
	用户主要通过修改 cuzthesis.tex 文件中的选项和调用不同的内容文件来定制论文。
	
	\subsection{文档类层}
	
	文档类层负责定义论文的基本格式和结构,主要包括:
	
	\begin{itemize}
		\item \textbf{cuzthesis.cls}:文档类定义文件,实现论文的基本格式设置,如页面布局、字体设置、章节样式等。
		
		\item \textbf{cuzthesis.cfg}:文档类配置文件,提供各种标签和常量的定义,便于国际化和定制。
	\end{itemize}
	
	文档类层是模板的核心,定义了论文的整体结构和外观。
	
	\subsection{功能模块层}
	
	功能模块层提供各种专门的功能支持,主要包括:
	
	\begin{itemize}
		\item \textbf{artratex.sty}:提供常用宏包和文档设定,如参考文献样式、文献引用样式、页眉页脚设定等。
		
		\item \textbf{artracom.sty}:提供自定义命令和宏定义,便于用户使用特定功能。
	\end{itemize}
	
	功能模块层采用选项机制,用户可以通过在 cuzthesis.tex 中设置不同的选项来启用或禁用特定功能。
	
	\subsection{内容模块层}
	
	内容模块层包含论文的实际内容,主要包括:
	
	\begin{itemize}
		\item \textbf{initialization.tex}:初始化论文的基本信息,如标题、作者、学院等。
		
		\item \textbf{各章节文件}:包含论文的具体内容,如绪论、文献综述、方法、结果、讨论等。
		
		\item \textbf{参考文献文件}:包含参考文献的数据和样式定义。
	\end{itemize}
	
	内容模块层是用户主要关注和修改的部分,用户通过编辑这些文件来完成论文的撰写。
	
	\section{关键技术实现}\label{sec:key-technologies}
	
	\subsection{文档类设计}
	
	cuzthesis 文档类基于 ctexbook 文档类开发,继承了 ctexbook 的基本功能,并进行了定制和扩展。主要技术实现包括:
	
	\begin{enumerate}
		\item \textbf{选项处理机制}:使用 LaTeX 的 keyval 机制处理文档类选项,支持多种选项组合。
		
		\item \textbf{页面布局设计}:使用 geometry 宏包精确控制页面尺寸、页边距和页眉页脚位置。
		
		\item \textbf{字体设置}:使用 fontspec 和 xeCJK 宏包设置中英文字体,支持不同操作系统的字体自动检测和替换。
		
		\item \textbf{章节样式定制}:使用 ctex 宏包的章节样式定制功能,实现符合浙传要求的章节标题格式。
	\end{enumerate}
	
	\subsection{参考文献管理}
	
	参考文献管理是学术论文的重要组成部分,cuzthesis 模板采用 BibTeX 系统进行参考文献管理,主要技术实现包括:
	
	\begin{enumerate}
		\item \textbf{参考文献样式}:使用符合国标 GB/T 7714-2015 的 BibTeX 样式文件(gbt7714-plain.bst 和 gbt7714-unsrt.bst)。
		
		\item \textbf{引用样式定制}:支持多种引用样式(numbers、super、authoryear、alpha),用户可根据需要选择。
		
		\item \textbf{多语言支持}:支持中英文混排的参考文献,自动处理不同语言的标点符号和排序规则。
	\end{enumerate}
	
	\subsection{浮动体处理}
	
	浮动体(图表、算法、代码等)的处理是论文排版的重要部分,cuzthesis 模板对浮动体进行了精心设计,主要技术实现包括:
	
	\begin{enumerate}
		\item \textbf{图表编号格式}:使用 caption 宏包定制图表编号格式,支持"章-序号"的编号方式。
		
		\item \textbf{多图排版}:使用 subcaption 宏包支持子图排版,便于展示多个相关图形。
		
		\item \textbf{算法环境}:使用 algorithm 和 algorithmic 宏包实现算法的排版。
		
		\item \textbf{代码高亮}:使用 minted 宏包实现代码的语法高亮,支持多种编程语言。
	\end{enumerate}
	
	\subsection{跨平台兼容性}
	
	为确保模板在不同操作系统和编译环境下都能正常工作,cuzthesis 模板采取了以下技术措施:
	
	\begin{enumerate}
		\item \textbf{字体自动检测}:根据不同操作系统自动检测并使用合适的字体,避免字体缺失问题。
		
		\item \textbf{编译引擎适配}:针对不同的 LaTeX 编译引擎(pdfLaTeX、XeLaTeX、LuaLaTeX)进行适配,确保兼容性。
		
		\item \textbf{路径处理}:使用相对路径引用文件,避免不同操作系统路径表示不同导致的问题。
		
		\item \textbf{编码统一}:统一使用 UTF-8 编码,确保在不同平台上文本显示一致。
	\end{enumerate}
	
	\section{实现细节}\label{sec:implementation-details}
	
	\subsection{页面布局实现}
	
	页面布局是论文格式的基础,cuzthesis 模板根据浙江传媒学院的要求,精确设置了页面尺寸、页边距和页眉页脚位置。具体实现代码如下:
	
	\begin{listing}[htbp]
		\caption{页面布局设置代码}
		\label{code:page-layout}
		\begin{texcode}
			\RequirePackage{geometry}
			\geometry{
				paper=a4paper,
				top=2.5cm, bottom=2.5cm,
				left=3cm, right=2cm,
				headheight=0.5cm, footskip=0.8cm
			}
		\end{texcode}
	\end{listing}
	
	此外,模板还根据不同的排版模式(单面打印、双面打印、印刷出版)提供了不同的页面布局设置,用户可以通过文档类选项进行选择。
	
	\subsection{字体设置实现}
	
	字体设置是中文论文排版的关键,cuzthesis 模板使用 fontspec 和 xeCJK 宏包设置中英文字体,并根据不同操作系统自动检测和使用合适的字体。具体实现代码如下:
	
	\begin{listing}[htbp]
		\caption{字体设置代码}
		\label{code:font-setting}
		\begin{texcode}
			% 设置英文字体
			\setmainfont{Times New Roman}
			\setsansfont{Arial}
			\setmonofont{Courier New}
			
			% 设置中文字体
			\setCJKmainfont[BoldFont={SimHei}, ItalicFont={KaiTi}]{SimSun}
			\setCJKsansfont{SimHei}
			\setCJKmonofont{FangSong}
		\end{texcode}
	\end{listing}
	
	模板还提供了字体替代机制,当首选字体不可用时,会自动使用备选字体,确保在不同环境下都能正常编译。
	
	\subsection{章节样式实现}
	
	章节样式是论文格式的重要组成部分,cuzthesis 模板使用 ctex 宏包的章节样式定制功能,实现了符合浙传要求的章节标题格式。具体实现代码如下:
	
	\begin{listing}[htbp]
		\caption{章节样式设置代码}
		\label{code:chapter-style}
		\begin{texcode}
			\ctexset{
				chapter = {
					format = \linespread{1.5}\zihao{4}\bfseries\raggedright,
					number = \arabic{chapter},
					name = {},
					aftername = \hskip 0.5em,
					beforeskip = 1.5ex,
					afterskip = 1.5ex,
				},
				section = {
					format = \linespread{1.5}\zihao{-4}\bfseries\raggedright,
					aftername = \hskip 0.5em,
					beforeskip = 0.4ex,
					afterskip = 0.4ex,
					indent = 2em,
				},
				% 其他章节级别设置...
			}
		\end{texcode}
	\end{listing}
	
	此外,模板还定义了自定义的章节环境(cuzchapter),用于实现特殊的章节格式需求。
	
	\subsection{参考文献实现}
	
	参考文献是学术论文的重要组成部分,cuzthesis 模板使用 natbib 宏包和符合国标的 BibTeX 样式文件实现参考文献的排版。具体实现代码如下:
	
	\begin{listing}[htbp]
		\caption{参考文献设置代码}
		\label{code:bibliography}
		\begin{texcode}
			\RequirePackage[sort&compress]{natbib}
			\bibliographystyle{bibliography/gbt7714-unsrt} % 顺序编码制
			% 或 \bibliographystyle{bibliography/gbt7714-plain} % 著者-出版年制
			
			% 设置参考文献格式
			\setlength{\bibsep}{0.5ex}
			\renewcommand{\bibfont}{\small}
		\end{texcode}
	\end{listing}
	
	模板支持多种引用样式,用户可以根据需要选择不同的引用方式。
	
	\section{测试与评估}\label{sec:testing-evaluation}
	
	\subsection{测试方法}
	
	为确保模板的质量和可靠性,我们采用了以下测试方法:
	
	\begin{enumerate}
		\item \textbf{功能测试}:测试模板的各项功能是否正常工作,包括编译、引用、图表生成等。
		
		\item \textbf{兼容性测试}:在不同操作系统(Windows、Linux、macOS)和不同编译环境(pdfLaTeX、XeLaTeX、LuaLaTeX)下测试模板的兼容性。
		
		\item \textbf{格式符合性测试}:检查生成的论文是否符合浙江传媒学院的格式要求。
		
		\item \textbf{用户体验测试}:邀请不同背景的用户使用模板,收集反馈意见。
	\end{enumerate}
	
	\subsection{测试结果}
	
	测试结果表明,cuzthesis 模板在功能、兼容性和格式符合性方面表现良好:
	
	\begin{itemize}
		\item \textbf{功能测试}:模板的各项功能都能正常工作,包括编译、引用、图表生成等。
		
		\item \textbf{兼容性测试}:模板在 Windows、Linux 和 macOS 系统上都能正常工作,支持 XeLaTeX 编译引擎,部分支持 pdfLaTeX 和 LuaLaTeX 编译引擎。
		
		\item \textbf{格式符合性测试}:生成的论文符合浙江传媒学院的格式要求,包括页面布局、字体设置、章节样式等。
		
		\item \textbf{用户体验测试}:用户反馈表明,模板易于使用,文档清晰,能够满足大多数用户的需求。
	\end{itemize}
	
	\subsection{性能评估}
	
	我们对模板的编译性能进行了评估,结果如下:
	
	\begin{table}[htbp]
		\caption{不同编译环境下的编译时间(秒)}
		\label{tab:compile-time}
		\centering
		\begin{tabular}{lccc}
			\toprule
			操作系统 & XeLaTeX & pdfLaTeX & LuaLaTeX \\
			\midrule
			Windows 10 & 8.5 & 5.2 & 12.3 \\
			Ubuntu 20.04 & 7.8 & 4.9 & 11.5 \\
			macOS Big Sur & 7.2 & 4.5 & 10.8 \\
			\bottomrule
		\end{tabular}
	\end{table}
	
	从表 \ref{tab:compile-time} 可以看出,pdfLaTeX 的编译速度最快,但对中文支持有限;XeLaTeX 的编译速度适中,且对中文支持最好;LuaLaTeX 的编译速度最慢,但功能最为强大。考虑到中文支持和功能需求,我们推荐使用 XeLaTeX 编译引擎。
	
	\section{小结}\label{sec:design-summary}
	
	本章详细介绍了 cuzthesis 模板的设计目标、系统架构、关键技术实现、实现细节以及测试与评估结果。通过分层架构设计和模块化实现,cuzthesis 模板实现了易用性、可扩展性和跨平台兼容性的目标,为浙江传媒学院的学生提供了一个高质量的论文排版工具。
	
	模板的设计和实现过程中,我们充分考虑了用户需求和使用场景,采用了多种技术手段确保模板的质量和可靠性。测试结果表明,模板能够满足浙江传媒学院毕业论文的格式要求,并提供良好的用户体验。
	
	未来,我们将继续完善模板,增加更多功能,提高兼容性,并根据用户反馈进行优化,使模板更好地服务于浙江传媒学院的学生。

\end{cuzchapter}
% 模板设计与实现
    \section{环境配置与安装}\label{sec:installation}

    在开始使用 cuzthesis 模板之前,需要先配置 \LaTeX{} 环境并安装必要的软件。本节将介绍环境配置和模板安装的步骤。

    \subsection{\LaTeX{} 环境配置}\label{sub:latex-environment}

    \LaTeX{} 是一种基于 \TeX{} 的排版系统,用于生成高质量的科技和学术文档。要使用 cuzthesis 模板,首先需要安装 \LaTeX{} 发行版。

    \begin{itemize}
        \item \textbf{Windows 系统}:推荐安装 TeX Live 或 MiKTeX。
            \begin{itemize}
                \item TeX Live:访问 \url{https://tug.org/texlive/} 下载并安装完整版 TeX Live。
                \item MiKTeX:访问 \url{https://miktex.org/} 下载并安装 MiKTeX。
            \end{itemize}

        \item \textbf{macOS 系统}:推荐安装 MacTeX。
            \begin{itemize}
                \item 访问 \url{https://tug.org/mactex/} 下载并安装 MacTeX。
                \item 也可以通过 Homebrew 安装:\verb|brew install --cask mactex|
            \end{itemize}

        \item \textbf{Linux 系统}:推荐安装 TeX Live。
            \begin{itemize}
                \item Ubuntu/Debian:\verb|sudo apt install texlive-full|
                \item Fedora:\verb|sudo dnf install texlive-scheme-full|
                \item Arch Linux:\verb|sudo pacman -S texlive-most texlive-lang|
            \end{itemize}
    \end{itemize}

    \subsection{编辑器选择}\label{sub:editor-choice}

    \LaTeX{} 文档可以使用任何文本编辑器编写,但使用专门的 \LaTeX{} 编辑器或配置好的代码编辑器可以提高工作效率。以下是一些推荐的编辑器:

    \begin{itemize}
        \item \textbf{VS Code}:搭配 \LaTeX{} Workshop 插件,提供语法高亮、自动补全、实时预览等功能。
        \item \textbf{TeXstudio}:集成开发环境,提供丰富的编辑功能和内置预览。
        \item \textbf{Overleaf}:基于网页的在线 \LaTeX{} 编辑器,支持协作编辑。
        \item \textbf{Texmaker/TeXworks}:轻量级编辑器,适合初学者使用。
    \end{itemize}

    \subsection{获取模板}\label{sub:get-template}

    cuzthesis 模板可以从 GitHub 仓库获取:

    \begin{itemize}
        \item 访问 \url{https://github.com/xiehao/CUZThesis} 下载最新版本的模板。
        \item 点击 "Code" 按钮,然后选择 "Download ZIP",或者使用 Git 克隆仓库:
        \begin{verbatim}
    git clone https://github.com/xiehao/CUZThesis.git
        \end{verbatim}
        \item 解压下载的文件到本地目录。
    \end{itemize}

    \begin{leftbar}
        \noindent\textbf{注意:}建议下载整个 cuzthesis 文件夹,而不是单独的文档类文件。cuzthesis 模板不仅提供了相应的类文件,还包括参考文献样式、示例图片等完成学位论文所需的所有组件。
    \end{leftbar}
\begin{cuzchapter}{使用指南与示例}{chap:guide}

    %---------------------------------------------------------------------------%
    %->> 使用指南章节写作指南
    %---------------------------------------------------------------------------%
    % 本章主要提供模板的使用指南和示例,帮助用户快速上手
    % 一个完整的使用指南章节应包含以下几个部分:
    % 1. 环境配置:介绍模板的安装和环境配置
    % 2. 基本用法:介绍模板的基本使用方法
    % 3. 高级功能:介绍模板的高级功能和定制方法
    % 4. 常见问题:解答用户可能遇到的常见问题
    % 5. 使用示例:提供各种排版元素的使用示例

    \section{模板概述}\label{sec:overview}

    cuzthesis 是为浙江传媒学院本科生设计的毕业论文 \LaTeX{} 模板,旨在帮助学生高效、规范地完成毕业论文的排版工作。本章将详细介绍模板的使用方法和各种功能,帮助用户快速上手并充分利用模板的各项特性。

    为方便使用及更好地展示 \LaTeX{} 排版的优秀特性,cuzthesis 对框架和文件体系进行了细致地处理,尽可能地对各个功能和板块进行了模块化和封装。对于初学者来说,众多的文件目录也许一开始让人觉得有些无所适从,但阅读完本章的使用说明后,会发现原来使用思路是简单而清晰的。

    当对 \LaTeX{} 有一定的认识和了解后,用户会发现其相对 Word 类排版系统极具吸引力的优秀特性。若是初学者,请不要退缩,请稍加尝试和坚持,以领略到 \LaTeX{} 的非凡魅力,并可以通过阅读相关资料如 \LaTeX{} Wikibook\citep{wikibook2014latex}来完善自己的使用知识。


    \section{快速入门}\label{sec:quickstart}

    本节将介绍如何快速开始使用 cuzthesis 模板撰写论文。

    \subsection{编译模板}

    首次使用模板时,建议先编译一次模板,确保环境配置正确。编译模板的方法有以下几种:

    \begin{itemize}
        \item \textbf{使用编译脚本}:
        \begin{itemize}
            \item \textbf{Windows}:双击运行 run.bat 脚本。
            \item \textbf{Linux 或 macOS}:在终端中执行以下命令:
            \begin{verbatim}
chmod +x ./scripts/run.sh
./scripts/run.sh xa cuzthesis
            \end{verbatim}
            其中,第一个参数 xa 指定编译模式(x 表示使用 \hologo{XeLaTeX},a 表示使用 \hologo{BibTeX}),第二个参数 cuzthesis 指定要编译的文件名(不包含 .tex 扩展名)。
        \end{itemize}

        \item \textbf{使用 \LaTeX{} 编辑器}:
        \begin{itemize}
            \item 打开 cuzthesis.tex 文件。
            \item 选择 \hologo{XeLaTeX} 编译引擎。
            \item 点击编译按钮或使用快捷键(VS Code 中为 Ctrl+Alt+B)。
        \end{itemize}
    \end{itemize}

    编译成功后,将在 cache 文件夹中生成 PDF 文档。如果编译过程中遇到问题,请参考本章后面的"常见问题"部分。

    \subsection{修改论文信息}

    编译成功后,下一步是修改论文的基本信息。打开 src/common/initialization.tex 文件,修改以下信息:

    \begin{itemize}
        \item 论文标题:修改 \verb|\title{}| 和 \verb|\englishtitle{}|
        \item 作者信息:修改 \verb|\author{}| 和 \verb|\authorid{}|
        \item 指导教师信息:修改 \verb|\advisor{}| 和 \verb|\advisortitle{}|
        \item 学院和专业信息:修改 \verb|\major{}|、\verb|\class{}| 和 \verb|\institute{}|
        \item 毕业年份:修改 \verb|\graduateyear{}|
    \end{itemize}

    修改完成后,重新编译模板,查看修改效果。

    \subsection{生成盲审版本}

    在提交论文进行盲审时,需要隐去作者、指导教师、学校等相关信息。cuzthesis 模板提供了盲审选项,只需在 cuzthesis.tex 文件中添加 blinded 选项即可:

    \begin{listing}[htbp]
        \caption{添加盲审选项}
        \label{code:blinded-option}
        \begin{texcode}
            % 正式版本
            \documentclass[singlesided,cuzhdr]{styles/cuzthesis}

            % 盲审版本
            \documentclass[singlesided,cuzhdr,blinded]{styles/cuzthesis}
        \end{texcode}
    \end{listing}

    添加盲审选项后,重新编译模板,生成的 PDF 文档将隐去作者、指导教师、学校等相关信息。

    \section{文档目录简介}\label{sec:directory}

    本部分主要介绍 cuzthesis 工程的目录结构,以帮助读者理解各部分的含义和用途。了解文档的组织结构对于高效使用模板和维护论文至关重要。

    \subsection{目录结构概览}\label{sub:directory-overview}

    cuzthesis 模板采用模块化的目录结构,将不同功能的文件分门别类地组织在不同的文件夹中,使得整个工程结构清晰、易于维护。总体结构如下所示:

    \begingroup
    \small\linespread{1}
    \begin{center}
        \begin{verbatim}
            ├── bibliography
            │   ├── gbt7714-plain.bst
            │   ├── gbt7714-unsrt.bst
            │   └── references.bib
            ├── cache
            │   └── ...
            ├── cuzassignment.tex
            ├── cuzopening.tex
            ├── cuzreview.tex
            ├── cuzthesis.tex
            ├── guidance.pdf
            ├── figures
            │   └── ...
            ├── latexmkrc
            ├── README.md
            ├── scripts
            │   ├── clean.sh
            │   ├── run.bat
            │   └── run.sh
            ├── src
            │   ├── assignment
            │   │   └── assignmentbody.tex
            │   ├── common
            │   │   └── initialization.tex
            │   ├── opening
            │   │   └── openingbody.tex
            │   ├── review
            │   │   └── reviewbody.tex
            │   └── thesis
            │       ├── backmatter
            │       │   ├── acknowledgement.tex
            │       │   └── appendices.tex
            │       ├── frontmatter
            │       │   └── abstracts.tex
            │       └── mainmatter
            │           ├── chapter_conclusions.tex
            │           ├── chapter_design.tex
            │           ├── chapter_guidance.tex
            │           ├── chapter_introduction.tex
            │           ├── chapter_literature.tex
            │           └── mainbody.tex
            └── styles
                ├── artracom.sty
                ├── artratex.sty
                ├── cuzassignment.cls
                ├── cuzbase.cls
                ├── cuzopening.cls
                ├── cuzreview.cls
                ├── cuzthesis.cfg
                └── cuzthesis.cls
        \end{verbatim}
    \end{center}
    \endgroup

    这种目录结构设计遵循了关注点分离的原则,将内容、样式和配置分开,使得用户可以专注于内容创作,而不必过多关注格式和排版细节。下面将详细介绍各个文件夹和关键文件的功能和用途。

    \subsection{主文档 (cuzthesis.tex)}\label{sub:cuzthesis}

    cuzthesis.tex 是整个论文的主文档,它设计和规划了论文的整体框架,通过阅读这个文件可以了解整个论文的结构和组织方式。主文档主要负责以下几个方面:

    \begin{enumerate}
        \item \textbf{文档类声明}:指定使用 cuzthesis 文档类及其选项,如单面/双面打印、页眉页脚样式等。
        \item \textbf{宏包加载}:加载论文所需的各种宏包,如 artratex.sty、artracom.sty 等。
        \item \textbf{文档结构组织}:通过 \verb|\input| 命令导入各个章节文件,组织论文的整体结构。
        \item \textbf{特殊页面生成}:生成封面、声明页、目录等特殊页面。
    \end{enumerate}

    一般情况下,用户只需要根据需要修改文档类选项和宏包选项,无需修改文档的整体结构。如果需要添加或删除章节,只需在主文档中添加或注释相应的 \verb|\input| 命令即可。

    \subsection{编译脚本 (scripts)}\label{sub:scripts}

    为了简化 \LaTeX{} 文档的编译过程,模板在 scripts 目录下提供了编译脚本,支持 Windows 和 Linux/macOS 系统:

    \begin{itemize}
        \item \textbf{Windows (run.bat)}:位于项目根目录,双击此批处理文件可以一键编译生成 PDF 文档。此脚本的主要目的是帮助不熟悉 \LaTeX{} 编译过程的初学者快速上手。\textbf{注意:请勿通过邮件传播和接收此脚本,以防范 DOS 脚本的潜在安全风险。}

        \item \textbf{Linux/macOS (run.sh)}:位于 scripts 目录下,在终端中运行以下命令:
              \begin{itemize}
                  \item \verb|./scripts/run.sh xa cuzthesis|:执行完整编译流程,生成最终 PDF 文档
                  \item \verb|./scripts/run.sh x cuzthesis|:执行快速编译模式,适用于内容修改但无新引用的情况
              \end{itemize}

              其中,第一个参数指定编译模式(x 表示使用 \hologo{XeLaTeX},a 表示使用 \hologo{BibTeX}),第二个参数指定要编译的文件名(不包含 .tex 扩展名)。
    \end{itemize}

    \subsubsection{编译模式说明}

    \begin{itemize}
        \item \textbf{完整编译}:执行 \verb|xelatex + bibtex + xelatex + xelatex| 编译流程,确保所有交叉引用、目录、参考文献等都正确生成。首次编译或添加新引用时应使用此模式。

        \item \textbf{快速编译}:仅执行一次 \verb|xelatex| 编译,适用于仅修改文本内容而未添加新引用的情况,可大幅减少编译时间。
    \end{itemize}

    对于使用集成开发环境(如 VS Code、TeXstudio 等)的用户,可以配置编辑器使用相同的编译流程,无需依赖这些脚本。

    \subsection{缓存目录 (cache)}\label{sub:cache}

    cache 文件夹是用于存放编译过程中生成的中间文件和最终 PDF 文档的目录。使用专门的缓存目录有以下几个优点:

    \begin{itemize}
        \item \textbf{保持工作空间整洁}:将大量的临时文件(如 .aux、.log、.toc 等)集中存放,避免它们与源文件混杂在一起,使项目目录结构更加清晰。

        \item \textbf{便于清理}:需要清理临时文件时,只需删除 cache 目录中的内容,而不必担心误删源文件。

        \item \textbf{提高工作效率}:整洁的工作环境有助于提高工作效率和心情,特别是在长期编写大型文档时。
    \end{itemize}

    需要注意的是,如果不使用提供的编译脚本,而是直接通过 \LaTeX{} 编译器或编辑器编译文档,则临时文件会生成在当前工作目录中,而不是 cache 目录中。在这种情况下,可以考虑使用编辑器的清理功能或手动清理临时文件。

    \subsection{样式文件 (styles)}\label{sub:styles}

    styles 文件夹包含 cuzthesis 文档类的定义文件和配置文件,这些文件控制论文的整体格式和样式。这部分主要由模板开发者维护,一般用户无需修改。但了解这些文件的功能有助于在需要时进行有针对性的定制。

    \begin{itemize}
        \item \textbf{cuzthesis.cls}:文档类定义文件,是整个模板的核心。它定义了论文的基本格式,包括页面布局、字体设置、章节样式、页眉页脚等。该文件基于标准的 \LaTeX{} 文档类(如 book、report 等)进行扩展,添加了符合浙江传媒学院学位论文要求的特定格式。

        \item \textbf{cuzthesis.cfg}:文档类配置文件,包含各种常量标签和配置信息,如论文类型名称、学位级别、学校名称等。将这些配置信息单独放在一个文件中,便于在不修改核心文档类的情况下进行定制。

        \item \textbf{artratex.sty}:常用宏包及文档设定文件,负责加载论文所需的各种宏包,并设置参考文献样式、文献引用样式、页眉页脚等。这些功能通常具有开关选项,可以在主文档中通过以下命令进行启用或禁用:

              \path{\usepackage[options]{artratex}}

              其中 options 可以是 numbers(顺序编码制)、super(上标顺序编码制)、authoryear(著者-出版年制)等。

        \item \textbf{artracom.sty}:自定义命令文件,包含各种自定义的命令和环境,如矢量符号、张量符号、特殊环境等。这是添加新命令和宏包的推荐位置,可以在不修改核心文档类的情况下扩展功能。
    \end{itemize}

    如果需要更新模板,通常只需替换这些样式文件即可,而不必修改论文的内容文件。对于有特殊排版需求的用户,可以在了解这些文件的基础上进行有针对性的修改,但建议在修改前备份原始文件,并详细记录所做的更改。

    \subsection{源文件目录 (src)}\label{sub:src}

    src 目录包含论文的所有源文件,是用户在撰写论文时主要关注和修改的位置。该目录采用模块化的结构,将不同类型的内容分别放在不同的子目录中,使得整个论文结构更加清晰。

    \subsubsection{公共文件 (common)}\label{subsub:common}

    common 目录包含论文的公共文件,主要是 initialization.tex,它负责初始化论文中的各种信息:

    \begin{itemize}
        \item \textbf{initialization.tex}:定义论文的基本信息,如中英文标题、作者姓名、学号、专业、学院、毕业年份等。这些信息只需设置一次,后续在论文的各个部分(如封面、声明页、页眉等)都可以直接调用,确保信息的一致性。
    \end{itemize}

    \subsubsection{论文内容 (thesis)}\label{subsub:thesis}

    thesis 目录包含论文的主体内容,按照论文的结构分为三个子目录:

    \begin{itemize}
        \item \textbf{frontmatter}:前置部分,包含摘要等内容
            \begin{itemize}
                \item \textbf{abstracts.tex}:中英文摘要,包括关键词
            \end{itemize}

        \item \textbf{mainmatter}:主体部分,包含论文的各个章节
            \begin{itemize}
                \item \textbf{mainbody.tex}:主体结构文件,用于组织和引用各个章节文件
                \item \textbf{chapter\_introduction.tex}:绪论章节
                \item \textbf{chapter\_literature.tex}:文献综述章节
                \item \textbf{chapter\_design.tex}:设计与实现章节
                \item \textbf{chapter\_guidance.tex}:使用指南章节
                \item \textbf{chapter\_conclusions.tex}:结论与展望章节
            \end{itemize}

        \item \textbf{backmatter}:后置部分,包含致谢和附录等内容
            \begin{itemize}
                \item \textbf{acknowledgement.tex}:致谢内容
                \item \textbf{appendices.tex}:附录内容,如源代码、数据表格等
            \end{itemize}
    \end{itemize}

    \begin{leftbar}
        \noindent\textbf{注意:}所有文件都必须采用 UTF-8 编码,否则编译后将出现乱码。在开始写作时,可以只在 mainbody.tex 中引用当前正在编写的章节,以加快编译速度;当论文完成后,再引用所有章节进行完整编译。
    \end{leftbar}

    \subsection{图片目录 (figures)}\label{sub:figures}

    figures 目录用于存放论文中所需的各种图片和媒体文件。合理组织和管理图片文件对于保持论文的整洁和提高工作效率非常重要。

    \subsubsection{支持的图片格式}\label{subsub:image-formats}

    模板支持以下几种常见的图片格式:
    \begin{itemize}
        \item \textbf{.pdf}:适用于矢量图,如流程图、框架图、模块图等
        \item \textbf{.png}:适用于位图,特别是需要透明背景的图片
        \item \textbf{.jpg}:适用于照片等不需要透明背景的位图
    \end{itemize}

    其中,cuz\_logo.png 是浙江传媒学院的校徽文件,用于创建论文封面。

    \subsubsection{图片管理建议}\label{subsub:image-management}

    \begin{leftbar}
        \noindent\textbf{图片格式选择建议:}
        \begin{itemize}
            \item 对于流程图、框架图、模块图等结构性图形,建议使用矢量格式(.pdf),以保证在任何缩放比例下都能清晰显示。
            \item 对于屏幕截图、界面展示等,建议使用 .png 格式,特别是当需要透明背景时。
            \item 对于照片等真实图像,可以使用 .jpg 格式以节省空间。
            \item 原则上,能用矢量图的尽量不要用位图,以保证论文的专业性和清晰度。
        \end{itemize}

        \noindent\textbf{图片组织建议:}
        \begin{itemize}
            \item 不建议为各章节图片创建子目录,即使图片众多,只要采用合理的命名规则,也能方便查找。
            \item 图片命名可以采用"章节\_内容\_序号"的方式,如"intro\_framework\_1.pdf",这样可以直观地了解图片的归属和内容。
            \item 对于需要多次修改的图片,建议保留原始文件(如 .ai、.psd、.vsdx 等)在单独的目录中,以便日后修改。
        \end{itemize}
    \end{leftbar}

    \subsection{参考文献目录 (bibliography)}\label{sub:bibliography}

    bibliography 目录用于存放参考文献相关的文件,包括参考文献数据库和样式文件。参考文献是学术论文的重要组成部分,正确管理和引用参考文献对于提高论文的学术性和可信度至关重要。

    \subsubsection{文件说明}\label{subsub:bibliography-files}

    \begin{itemize}
        \item \textbf{references.bib}:参考文献数据库文件,用于存储所有引用文献的详细信息,如作者、标题、出版年份、期刊名称等。这是一个纯文本文件,采用 BibTeX 格式,可以使用文本编辑器或专门的参考文献管理软件(如 JabRef、Zotero、Mendeley 等)进行编辑。

        \item \textbf{gbt7714-plain.bst}:符合 GB/T 7714-2015 国标的参考文献样式文件,用于生成"著者-出版年"制的参考文献列表。

        \item \textbf{gbt7714-unsrt.bst}:符合 GB/T 7714-2015 国标的参考文献样式文件,用于生成按引用顺序排列的参考文献列表。
    \end{itemize}

    这些样式文件由 \href{https://github.com/zepinglee/gbt7714-bibtex-style}{zepinglee} 开发,满足最新的中国国家标准 GB/T 7714-2015《信息与文献 参考文献著录规则》的要求。关于参考文献样式的详细问题,请查阅开发者提供的文档,并建议适当追踪其更新。

    \subsubsection{参考文献管理建议}\label{subsub:bibliography-management}

    \begin{leftbar}
        \noindent\textbf{参考文献收集与管理:}
        \begin{itemize}
            \item 使用专业的参考文献管理软件(如 JabRef、Zotero、Mendeley 等)可以大大简化参考文献的收集、组织和引用过程。
            \item 从学术搜索引擎(如 Google Scholar、百度学术等)导出 BibTeX 格式的引用信息,可以避免手动输入错误。
            \item 定期备份 references.bib 文件,以防数据丢失。
        \end{itemize}

        \noindent\textbf{BibTeX 条目命名规范:}
        \begin{itemize}
            \item 英文文献建议使用"第一作者姓氏+年份+首词"的格式,如"smith2020analysis"。
            \item 中文文献建议使用"第一作者拼音+年份+首词拼音"的格式,如"zhang2019yanjiu"。
            \item 保持命名的一致性和可读性,避免使用特殊字符和空格。
        \end{itemize}
    \end{leftbar}

    \section{排版元素}\label{sec:elements}

    学位论文中的排版元素有很多,本模板无法逐一介绍,只就公式、图表等几种常用的排
    版元素的用法及注意事项简要说明如下,详细用法请参考相应资料。

    \subsection{数学公式}\label{sub:equations}

    比如Navier-Stokes方程,如式\eqref{eq:ns}和式\eqref{eq:ns-}所示:
    \begin{equation}
        \label{eq:ns}
        \begin{cases}
            \frac{\partial \rho}{\partial t} + \nabla\cdot(\rho\Vector{V}) = 0 \ \mathrm{times\ font\ test}                                            \\
            \frac{\partial (\rho\Vector{V})}{\partial t} + \nabla\cdot(\rho\Vector{V}\Vector{V}) = \nabla\cdot\Tensor{\sigma} \ \text{times font test} \\
            \frac{\partial (\rho E)}{\partial t} + \nabla\cdot(\rho E\Vector{V}) = \nabla\cdot(k\nabla T) + \nabla\cdot(\Tensor{\sigma}\cdot\Vector{V})
        \end{cases}
    \end{equation}
    \begin{equation}
        \label{eq:ns-}
        \frac{\partial }{\partial t}\int\limits_{\Omega} u \, \mathrm{d}\Omega + \int\limits_{S} \unitVector{n}\cdot(u\Vector{V}) \, \mathrm{d}S = \dot{\phi}
    \end{equation}

    数学公式常用命令请见
    \href{https://en.wikibooks.org/wiki/LaTeX/Mathematics}{WiKibook
        Mathematics}。 artracom.sty中对一些常用数据类型如矢量矩阵等进行了封装,这样
    的好处是如有一天需要修改矢量的显示形式,只需单独修改artracom.sty中的矢量定义
    即可实现全文档的修改。

    \subsection{表格}\label{sub:tables}

    请见表~\ref{tab:sample}。制表的更多范例,请见
    \href{https://en.wikibooks.org/wiki/LaTeX/Tables}{WiKibook Tables}。
    \begin{table}[!htbp]
        \caption[样表]{这是一个样表。}
        \label{tab:sample}
        \centering
        \footnotesize% fontsize
        \setlength{\tabcolsep}{4pt}% column separation
        \renewcommand{\arraystretch}{1.2}%row space
        \begin{tabular}{lcccccccc}
            \hline
            Row number & \multicolumn{8}{c}{This is a multicolumn}                                     \\
            %\cline{2-9}% partial hline from column i to column j
            \hline
            Row 1      & $1$                                       & $2$ & $4$ & $5$ & $6$ & $7$ & $8$ \\
            Row 2      & $1$                                       & $2$ & $4$ & $5$ & $6$ & $7$ & $8$ \\
            Row 3      & $1$                                       & $2$ & $4$ & $5$ & $6$ & $7$ & $8$ \\
            Row 4      & $1$                                       & $2$ & $4$ & $5$ & $6$ & $7$ & $8$ \\
            \hline
        \end{tabular}
    \end{table}

    \subsection{图片}\label{sub:images}

    一图胜千言,图片的插入在论文中往往能起到点睛的作用。在插入图片时,须在正文中
    加以引用,并配上相应的解释说明。同时,应将代码放置在引用处的后方合适位置(勿
    相距甚远)。论文中图片的插入通常分为单图和多图,下面分别加以介绍:

    单图插入:假设插入名为\verb|cuz_logo.png|(后缀可以为.jpg、.png、.pdf,下
    同)的图片,其效果如图~\ref{fig:cuz_logo}。注意,应在图的下方给出图例
    (\verb|\caption|)。
    \begin{figure}[h]
        \centering
        \includegraphics[width=0.5\textwidth]{cuz_logo}
        \caption[浙传校徽]{浙传校徽,同时测试一下一个很长的标题,比如这真的是一个很长很长很长很长很长很长很长很长的标题。}
        \label{fig:cuz_logo}
    \end{figure}

    若插图的空白区域过大,以图片\verb|shock_cyn|为例,自动裁剪如图
    ~\ref{fig:shock_cyn}。
    \begin{figure}[h]
        \centering
        % trim option's parameter order: left bottom right top
        \includegraphics[trim = 30mm 0mm 30mm 0mm, clip, width=0.40\textwidth]{shock_cyn}
        \caption[激波圆柱作用]{激波圆柱作用。}
        \label{fig:shock_cyn}
    \end{figure}

    多图的插入如图~\ref{fig:oaspl},多图不应在子图中给文本子标题,只要给序号,并
    在主标题中进行引用说明。
    \begin{figure}[h]
        \centering
        \begin{subfigure}[b]{0.35\textwidth}
            \includegraphics[width=\textwidth]{oaspl_a}
            \caption{}
            \label{fig:oaspl_a}
        \end{subfigure}%
        ~% add desired spacing
        \begin{subfigure}[b]{0.35\textwidth}
            \includegraphics[width=\textwidth]{oaspl_b}
            \caption{}
            \label{fig:oaspl_b}
        \end{subfigure}
        \begin{subfigure}[b]{0.35\textwidth}
            \includegraphics[width=\textwidth]{oaspl_c}
            \caption{}
            \label{fig:oaspl_c}
        \end{subfigure}%
        ~% add desired spacing
        \begin{subfigure}[b]{0.35\textwidth}
            \includegraphics[width=\textwidth]{oaspl_d}
            \caption{}
            \label{fig:oaspl_d}
        \end{subfigure}
        \caption[总声压级]{总声压级。(a) 这是子图说明信息,(b) 这是子图说明信息,(c) 这是子图说明信息,(d) 这是子图说明信息。}
        \label{fig:oaspl}
    \end{figure}

    \subsection{算法}\label{sub:algorithms}

    算法环境在浙传毕设论文官方Word模板中未作明确要求,故暂采用通用样式,如算法
    \ref{alg:euclid}所示,详细使用方法请参见文档
    \href{https://ctan.org/pkg/algorithmicx?lang=en}{algorithmicx}。

    \begin{algorithm}[h]
        \small
        \caption{Euclid算法}\label{alg:euclid}
        \begin{algorithmic}[1]
            \Procedure{Euclid}{$a,b$}\Comment{$a$与$b$的最大公约数}
            \State $r\gets a\bmod b$
            \While{$r\not=0$}\Comment{若$r$为0则可跳出循环并返回答案}
            \State $a\gets b$
            \State $b\gets r$
            \State $r\gets a\bmod b$
            \EndWhile\label{euclidendwhile}
            \State \textbf{return} $b$\Comment{最大公约数为$b$}
            \EndProcedure
        \end{algorithmic}
    \end{algorithm}

    \subsection{代码段}\label{sub:listings}

    在正文中引用一段代码,可使用lstlisting设置代码环境。本模板的代码环境默认配置
    在artratex.sty,搜索关键字“\verb|\lstset|”即可找到相应配置。

    观察代码段~\ref{code:samp-code},结合前述图表设置,试图理解代码环境的编写。

    % \begin{lstlisting}[
    %         language=C++,
    %         label=code:samp-code,
    %         caption=一段Chromium的源代码
    %     ]
    %     // Start tasks to take all the threads and block them.
    %     const int kNumBlockTasks = static_cast<int>(kNumWorkerThreads);
    %     for (int i = 0; i < kNumBlockTasks; ++i) {
    %         EXPECT_TRUE(pool()->PostWorkerTask(
    %             FROM_HERE,
    %             base::Bind(&TestTracker::BlockTask, tracker(), i, &blocker)));
    %     }
    %     tracker()->WaitUntilTasksBlocked(kNumWorkerThreads);

    %     // Setup to open the floodgates from within Shutdown().
    %     SetWillWaitForShutdownCallback(
    %         base::Bind(&TestTracker::PostBlockingTaskThenUnblockThreads,
    %                     scoped_refptr<TestTracker>(tracker()), pool(), &blocker,
    %                     kNumWorkerThreads));
    %     pool()->Shutdown(kNumWorkerThreads + 1);

    %     // Ensure that the correct number of tasks actually got run.
    %     tracker()->WaitUntilTasksComplete(static_cast<size_t>(kNumWorkerThreads + 1));
    %     tracker()->ClearCompleteSequence();
    % \end{lstlisting}
    \begin{listing}[H]
        \centering
        \caption{一段Chromium的源代码}
        \label{code:samp-code}
        \begin{cppcode}
            // Start tasks to take all the threads and block them.
            const int kNumBlockTasks = static_cast<int>(kNumWorkerThreads);
            for (int i = 0; i < kNumBlockTasks; ++i) {
                EXPECT_TRUE(pool()->PostWorkerTask(
                    FROM_HERE,
                    base::Bind(&TestTracker::BlockTask, tracker(), i, &blocker)));
            }
            tracker()->WaitUntilTasksBlocked(kNumWorkerThreads);

            // Setup to open the floodgates from within Shutdown().
            SetWillWaitForShutdownCallback(
                base::Bind(&TestTracker::PostBlockingTaskThenUnblockThreads,
                            scoped_refptr<TestTracker>(tracker()), pool(), &blocker,
                            kNumWorkerThreads));
            pool()->Shutdown(kNumWorkerThreads + 1);

            // Ensure that the correct number of tasks actually got run.
            tracker()->WaitUntilTasksComplete(static_cast<size_t>(kNumWorkerThreads + 1));
            tracker()->ClearCompleteSequence();
        \end{cppcode}
    \end{listing}

    引用一两行代码,可以直接使用\texttt{verbatim}环境完成;若想调整环境中字体大
    小,可先用\verb|\begingroup|和\verb|\endgroup|将其包住,后加入字体大小命令。
    注意,此环境不会采取任何主动断行策略。

    \begingroup
    \small
    \begin{verbatim}
Error: Command failed: /bin/sh -c rsync -arvq --exclude cache
--exclude .git
    \end{verbatim}
    \endgroup

    \begin{leftbar}
        \noindent\textbf{建议:}原则上,论文正文中应尽可能少出现工程代码片段,建
        议每段代码不超过一页(半页以内尤佳),并在正文中配有相应解释说明;超过一
        页的代码片段可拆分成多个模块(函数)分别列出并解释,若实在无法拆分,可将
        其放到附录中。
    \end{leftbar}

    \subsection{参考文献引用}\label{sub:references}

    参考文献引用过程以实例进行介绍,假设需要引用名为``Document Preparation
    System''的文献,步骤如下:
    \begin{enumerate}
        \item 使用Google Scholar搜索Document Preparation System,在目标条目下点
              击Cite,展开后选择Import into BibTeX打开此文章的\hologo{BibTeX}索
              引信息,将它们copy添加到references.bib文件中(此文件位于
              bibliography文件夹下)。
        \item 索引第一行 \verb|@article{lamport1986document,|中
              \verb|lamport1986document|即为此文献的label (\textbf{中文文献也必
                  须使用英文label},一般遵照:姓氏拼音+年份+标题第一字拼音的格式),
              想要在论文中索引此文献,有两种索引类型:
              \begin{itemize}
                  \item 文本类型:\verb|\citet{lamport1986document}|,正如此处所
                        示\citet{lamport1986document};
                  \item 括号类型:\verb|\citep{lamport1986document}|。正如此处所
                        示\citep{lamport1986document}。
              \end{itemize}
              \textbf{多文献索引用须用英文逗号隔开}:
              \begin{itemize}
                  \item \verb|\citep{lamport1986document, chu2004tushu, chen2005zhulu}|,
                        正如此处所示\citep{lamport1986document, chu2004tushu, chen2005zhulu}。
              \end{itemize}
    \end{enumerate}

    更多例子如:\citet{walls2013drought}根据...的研究,首次提出...。其中关于
    ...\citep{walls2013drought},是当前中国...得到迅速发展的研究领域
    \citep{chen1980zhongguo}。引用同一著者在同一年份出版的多篇文献时,在出版年份
    之后用英文小写字母区别,如:\citep{yuan2012lana, yuan2012lanb,
        yuan2012lanc}。同一处引用多篇文献时,按出版年份由近及远依次标注,中间用分号
    分开,例如\citep{chen1980zhongguo, stamerjohanns2009mathml, hls2012jinji,
        niu2013zonghe}。

    使用著者-出版年制(authoryear)式参考文献样式时,中文文献必须在
    \hologo{BibTeX}索引信息的\textbf{key} 域(请参考references.bib文件)填写作者
    姓名的拼音,才能使得文献列表按照拼音排序。参考文献表中的条目(不排序号),先
    按语种分类排列,语种顺序是:中文、日文、英文、俄文、其他文种。然后,中文按汉
    语拼音字母顺序排列,日文按第一著者的姓氏笔画排序,西文和 俄文按第一著者姓氏
    首字母顺序排列。如中\citep{niu2013zonghe}、日\citep{Bohan1928}、英
    \citep{stamerjohanns2009mathml}、俄\citep{Dubrovin1906}。

    不同文献样式和引用样式,如著者-出版年制(authoryear)、顺序编码制
    (numbers)、上标顺序编码制(super)可在cuzthesis.tex中修改调用artratex.sty
    的参数实现,如:
    \begin{itemize}
        \item \verb+\usepackage[numbers]{artratex}+ $\%$ 文本: Jones [1]; 括号: [1]
        \item \verb+\usepackage[super]{artratex}+ $\%$ 文本: Jones 上标[1]; 括号: 上标[1]
        \item \verb+\usepackage[authoryear]{artratex}+ $\%$ 文本: Jones (1995); 括号: (Jones, 1995)
        \item \verb+\usepackage[alpha]{artratex}+ $\%$ 文本: 不可用; 括号: [Jon95]
    \end{itemize}

    当前文档的默认参考文献样式为\textbf{super}。在该模式下,若希望在特定位置将上
    标改为嵌入式标,可使用:

    \begin{itemize}
        \item 文本类型:\verb|\citetns{lamport1986document,chen2005zhulu}|,正如此处
              所示\citetns{lamport1986document,chen2005zhulu};
        \item 括号类型:\verb|\citepns{lamport1986document,chen2005zhulu}|;正如此处
              所示\citepns{lamport1986document,chen2005zhulu}。
    \end{itemize}

    参考文献索引更为详细的信息,请见
    \href{https://github.com/zepinglee/gbt7714-bibtex-style}{zepinglee} 和
    \href{https://en.wikibooks.org/wiki/LaTeX/Bibliography_Management}{WiKibook
        Bibliography}。

    % \nocite{*}

    \section{现代 \LaTeX{} 功能}\label{sec:modern-latex}

    \LaTeX{} 生态系统不断发展,提供了许多现代化的功能和宏包,可以大大提高论文的质量和写作效率。本节介绍一些现代 \LaTeX{} 功能的使用示例。

    \subsection{TikZ 绘图}\label{sub:tikz}

    TikZ 是一个强大的绘图工具,可以创建高质量的矢量图形。以下是一个简单的 TikZ 绘图示例,展示了一个流程图:

    \begin{figure}[htbp]
        \centering
        \begin{tikzpicture}[
            node distance=2cm,
            box/.style={rectangle, draw, text width=3cm, text centered, minimum height=1cm, rounded corners},
            arrow/.style={thick, ->, >=stealth}
            ]

            % 定义节点
            \node[box] (start) {开始};
            \node[box, below of=start] (input) {输入数据};
            \node[box, below of=input] (process) {处理数据};
            \node[box, below of=process] (output) {输出结果};
            \node[box, below of=output] (end) {结束};

            % 连接节点
            \draw[arrow] (start) -- (input);
            \draw[arrow] (input) -- (process);
            \draw[arrow] (process) -- (output);
            \draw[arrow] (output) -- (end);

        \end{tikzpicture}
        \caption{使用 TikZ 创建的简单流程图}
        \label{fig:tikz-flowchart}
    \end{figure}

    TikZ 还可以绘制更复杂的图形,如数学图表、网络拓扑图、状态机等。详细用法请参考 TikZ 文档。

    \subsection{Beamer 演示文稿}\label{sub:beamer}

    Beamer 是一个用于创建演示文稿的 \LaTeX{} 文档类,可以生成高质量的 PDF 幻灯片。以下是一个简单的 Beamer 演示文稿示例:

    \begin{listing}[htbp]
        \caption{Beamer 演示文稿示例}
        \label{code:beamer-example}
        \begin{texcode}
            \documentclass{beamer}
            \usetheme{Madrid}
            \usecolortheme{beaver}

            \title{\LaTeX{} 演示文稿}
            \author{张三}
            \institute{浙江传媒学院}
            \date{\today}

            \begin{document}

            \begin{frame}
                \titlepage
            \end{frame}

            \begin{frame}{目录}
                \tableofcontents
            \end{frame}

            \section{引言}

            \begin{frame}{引言}
                \begin{itemize}
                    \item 第一点
                    \item 第二点
                    \item 第三点
                \end{itemize}
            \end{frame}

            \section{方法}

            \begin{frame}{方法}
                \begin{enumerate}
                    \item 步骤一
                    \item 步骤二
                    \item 步骤三
                \end{enumerate}
            \end{frame}

            \end{document}
        \end{texcode}
    \end{listing}

    Beamer 提供了多种主题和颜色方案,可以根据需要进行定制。

    \subsection{Bib\LaTeX{} 参考文献管理}\label{sub:biblatex}

    Bib\LaTeX{} 是一个现代化的参考文献管理工具,比传统的 \hologo{BibTeX} 提供了更多的功能和更灵活的定制选项。以下是使用 Bib\LaTeX{} 的示例:

    \begin{listing}[htbp]
        \caption{Bib\LaTeX{} 使用示例}
        \label{code:biblatex-example}
        \begin{texcode}
            % 在导言区加载 biblatex 宏包
            \usepackage[
                backend=biber,
                style=gb7714-2015,
                sorting=nyt,
                giveninits=true
            ]{biblatex}

            % 指定参考文献数据库
            \addbibresource{references.bib}

            % 在文档中引用文献
            \cite{lamport1986document}
            \textcite{chen1980zhongguo}
            \parencite{stamerjohanns2009mathml}

            % 在文档末尾打印参考文献列表
            \printbibliography[title=参考文献]
        \end{texcode}
    \end{listing}

    Bib\LaTeX{} 支持多种引用样式和参考文献格式,可以满足不同学科和期刊的要求。

    \subsection{Markdown 与 \LaTeX{} 集成}\label{sub:markdown}

    对于不熟悉 \LaTeX{} 的用户,可以使用 Markdown 编写内容,然后转换为 \LaTeX{}。Pandoc 是一个强大的文档转换工具,可以将 Markdown 转换为 \LaTeX{}。以下是一个简单的 Markdown 示例:

    \begin{listing}[htbp]
        \caption{Markdown 示例}
        \label{code:markdown-example}
        \begin{texcode}
            # 标题

            ## 小标题

            这是一段普通文本,包含**粗体**和*斜体*。

            - 列表项 1
            - 列表项 2
            - 列表项 3

            1. 有序列表项 1
            2. 有序列表项 2
            3. 有序列表项 3

            > 这是一段引用文本。

            ```python
            def hello_world():
                print("Hello, World!")
            ```

            [链接文本](https://example.com)

            ![图片描述](image.png)
        \end{texcode}
    \end{listing}

    使用 Pandoc 将 Markdown 转换为 \LaTeX{} 的命令如下:

    \begin{verbatim}
pandoc -f markdown -t latex -o output.tex input.md
    \end{verbatim}

    \section{常见使用问题}\label{sec:qa}

    \begin{itemize}
        \item 模板每次发布前,都已在Windows,Linux,macOS系统上测试通过。下载模
              板后,若编译出现错误,则请参考国科大模板附带的
              \href{https://github.com/mohuangrui/ucasthesis/wiki}{\LaTeX{}知识
                  小站} 中的
              \href{https://github.com/mohuangrui/ucasthesis/wiki/%E7%BC%96%E8%AF%91%E6%8C%87%E5%8D%97}{编
                  译指南}。
        \item 模板文档的编码为UTF-8编码。所有文件都必须采用UTF-8编码,否则编译后
              生成的文档将出现乱码文本。若出现文本编辑器无法打开文档或打开文档乱
              码的问题,请检查编辑器对UTF-8编码的支持。
        \item 推荐选择 \hologo{XeLaTeX} 编译引擎编译中文文档。编译脚本的默认设定为
              \hologo{XeLaTeX} 编译引擎。你也可以选择不使用脚本编译,如直接使用
              \LaTeX{} 文本编辑器编译。注:\LaTeX{} 文本编辑器编译的默认设定为
              \hologo{pdfLaTeX} 编译引擎,若选择 \hologo{XeLaTeX} 编译引擎,请进入
              下拉菜单选择。为正确生成引用链接,需要进行全编译。由于
              \hologo{LuaLaTeX} 编译引擎尚不成熟,故暂不推荐。
        \item VS Code 中关于 \LaTeX{} 方面建议安装的插件:
              \begin{itemize}
                  \item \LaTeX{} Workshop:提供了绝大多数 \LaTeX{} 的
                        辅助功能;
                  \item Rewrap:可使用\verb|Alt+Q|进行硬换行(即自动重排段落使得
                        每行不超过指定宽度)。
              \end{itemize}
              其他一些有用的插件有:
              \begin{itemize}
                  \item Git Graph;以更形象的方式查看Git提交记录,并可做出一些简
                        单Git操作;
                  \item gitignore:对.gitignore文件进行操作;
                  \item Markdown Preview Enhanced:提供了Markdown语法支持与预
                        览。
              \end{itemize}
        \item 设置文档样式: 在artratex.sty中搜索关键字定位相应命令,然后修改:
              \begin{itemize}
                  \item 正文行距:启用和设置 \verb|\linespread{1.5}|,默认1.5倍
                        行距。
                  \item 参考文献行距:修改 \verb|\setlength{\bibsep}{0.0ex}|
                  \item 目录显示级数:修改 \verb|\setcounter{tocdepth}{2}|
                  \item 文档超链接的颜色及其显示:修改 \verb|\hypersetup|
              \end{itemize}
        \item 文档内字体切换方法:
              \begin{itemize}
                  \item 宋体:浙传论文模板cuzthesis 或 \textrm{浙传论文模板
                            cuzthesis}
                  \item 粗宋体:{\bfseries 浙传论文模板cuzthesis} 或 \textbf{浙
                            传论文模板cuzthesis}
                  \item 黑体:{\sffamily 浙传论文模板cuzthesis} 或 \textsf{浙传
                            论文模板cuzthesis}
                  \item 粗黑体:{\bfseries\sffamily 浙传论文模板cuzthesis} 或
                        \textsf{\bfseries 浙传论文模板cuzthesis}
                  \item 仿宋:{\ttfamily 浙传论文模板cuzthesis} 或 \texttt{浙传
                            论文模板cuzthesis}
                  \item 粗仿宋:{\bfseries\ttfamily 浙传论文模板cuzthesis} 或
                        \texttt{\bfseries 浙传论文模板cuzthesis}
                  \item 楷体:{\itshape 浙传论文模板cuzthesis} 或 \textit{浙传论
                            文模板cuzthesis}
                  \item 粗楷体:{\bfseries\itshape 浙传论文模板cuzthesis} 或
                        \textit{\bfseries 浙传论文模板cuzthesis}
              \end{itemize}
    \end{itemize}

\end{cuzchapter}
% 使用指南
\chapter{总结与展望}\label{chap:conclusions}

本文主要介绍了浙传\hologo{LaTeX}模板cuzthesis的一些基本用法与注意事项,同时也对
论文各部分的写法做了简要说明并提出了一些意见与建议。相信读者通过阅读本文档,当可
对浙传毕业论文的撰写(包括内容与格式)有一个大概认识。

诚然,不足之处在所难免:模板类中一些设计尚存在不甚合理之处,适用情况亦不够普遍。
但相信随着\hologo{LaTeX}知识的增长,上述缺憾可逐一弥补。

\begin{leftbar}
    \noindent\textbf{建议:}该部分应对全文做出总结与概括,抓住重点、简明扼要;在
    此基础上,列出不足之处,并简要提出可能的改进想法与未来展望。
\end{leftbar}% 结论
%---------------------------------------------------------------------------%


%->> 后置部分:参考文献、致谢、附录
%---------------------------------------------------------------------------%
% 添加参考文献到目录和书签
\intotoc{\bibname}

% 生成参考文献
% 参考文献数据库文件为 bibliography/references.bib
% 如需添加新的参考文献,请编辑该文件
% 如需调整参考文献的行距,可取消下面的注释
% \begingroup
% \linespread{1.2}\selectfont
\bibliography{bibliography/references}
% \endgroup

% 导入致谢部分
% 在 src/thesis/backmatter/acknowledgement.tex 中编写致谢内容
% 致谢应简明扼要,对支持和帮助过论文工作的人员和单位表示感谢
\begin{acknowledgement}
    感激casthesis作者吴凌云学长,gbt7714-bibtex-style
    开发者zepinglee,和ctex众多开发者们。若没有他们的辛勤付出和非凡工作,\LaTeX{}菜鸟的我是无法完成此国科大学位论文\LaTeX{}模板ucasthesis的。在\LaTeX{}中的一点一滴的成长源于开源社区的众多优秀资料和教程,在此对所有\LaTeX{}社区的贡献者表示感谢!
    
    ucasthesis国科大学位论文\LaTeX{}模板的最终成型离不开以霍明虹老师和丁云云老师为代表的国科大学位办公室老师们制定的官方指导文件和众多ucasthesis用户的热心测试和耐心反馈,在此对他们的认真付出表示感谢。特别对国科大的赵永明同学的众多有效反馈意见和建议表示感谢,对国科大本科部的陆晴老师和本科部学位办的丁云云老师的细致审核和建议表示感谢。谢谢大家的共同努力和支持,让ucasthesis为国科大学子使用\LaTeX{}撰写学位论文提供便利和高效这一目标成为可能。
\end{acknowledgement}

% 设置为附录模式(章节编号变为"附录A"等格式)
\appendix

% 导入附录内容(若无则可注释掉)
% 在 src/thesis/backmatter/appendices.tex 中编写附录内容
% 附录可包含对正文的补充说明、源代码、数据表格等内容
% \chapter{中国科学院大学学位论文撰写要求}
\begin{appendices}\label{sec:appendices}

	\section*{论文无附录者无需附录部分}

	\section*{测试公式编号} \label{sec:testmath}

	参见式\eqref{eq:appedns}与式\eqref{eq:2}:

	\begin{equation} \label{eq:appedns}
		\begin{cases}
			\frac{\partial \rho}{\partial t} + \nabla\cdot(\rho\Vector{V}) = 0 \ \mathrm{times\ font\ test}                                            \\
			\frac{\partial (\rho\Vector{V})}{\partial t} + \nabla\cdot(\rho\Vector{V}\Vector{V}) = \nabla\cdot\Tensor{\sigma} \ \text{times font test} \\
			\frac{\partial (\rho E)}{\partial t} + \nabla\cdot(\rho E\Vector{V}) = \nabla\cdot(k\nabla T) + \nabla\cdot(\Tensor{\sigma}\cdot\Vector{V})
		\end{cases}
	\end{equation}
	\begin{equation} \label{eq:2}
		\frac{\partial }{\partial t}\int\limits_{\Omega} u \, \mathrm{d}\Omega + \int\limits_{S} \unitVector{n}\cdot(u\Vector{V}) \, \mathrm{d}S = \dot{\phi}
	\end{equation}

	\section*{测试代码段编号} \label{sec:testlistings}

	参见代码段
	\ref{code:samp-code-c}、\ref{code:samp-code-cpp}、\ref{code:samp-code-java}、\ref{code:samp-code-csharp}、
	\ref{code:samp-code-python}与\ref{code:samp-code-latex}:

    \begin{listing}[H]
        \centering
        \caption{一段简单得不能再简单的C代码}
        \label{code:samp-code-c}
        \begin{ccode}
            /* 一段简单得不能再简单的C代码 */
            #include <stdio.h>

            int main(int argc, char const **argv) {
                printf("Hello CUZThesis!!\n");
                return 0;
            }
        \end{ccode}
    \end{listing}

    \begin{listing}[H]
        \centering
        \caption{一段简单得不能再简单的C++代码}
        \label{code:samp-code-cpp}
        \begin{cppcode}
            /* 一段简单得不能再简单的C++代码 */
            #include <iostream>

            int main(int argc, char const **argv) {
                std::cout << "Hello CUZThesis!!" << std::endl;
                return 0;
            }
        \end{cppcode}   
    \end{listing}

    \begin{listing}[H]
        \centering
        \caption{一段简单得不能再简单的Java代码}
        \label{code:samp-code-java}
        \begin{javacode}
            // 一段简单得不能再简单的Java代码
            class HelloCUZThesis {
                public static void main(String[] args) {
                    System.out.println("Hello CUZThesis!!");
                }
            }
        \end{javacode}
    \end{listing}

    \begin{listing}[H]
        \centering
        \caption{一段简单得不能再简单的C\#代码}
        \label{code:samp-code-csharp}
        \begin{csharpcode}
            // 一段简单得不能再简单的C#代码
            class HelloCUZThesis {
                public static void Main(string[] args) {
                    Console.WriteLine("Hello CUZThesis!!");
                }
            }
        \end{csharpcode}
    \end{listing}

    \begin{listing}[H]
        \centering
        \caption{一段简单得不能再简单的Python代码}
        \label{code:samp-code-python}
        \begin{pythoncode}
            # 一段简单得不能再简单的Python代码
            print("Hello CUZThesis!!")
        \end{pythoncode}
    \end{listing}

    \begin{listing}[H]
        \centering
        \caption{一段简单得不能再简单的\hologo{LaTeX}代码}
        \label{code:samp-code-latex}
        \begin{texcode}[texcomments]
            % 一段简单得不能再简单的\hologo{LaTeX}代码
            \documentclass{article}

            \begin{document}    
                Hello CUZThesis!!
            \end{document}
        \end{texcode}
    \end{listing}

	\section*{测试生僻字} \label{sec:testcharacters}

	霜蟾盥薇曜灵霜颸妙鬘虚霩淩澌菀枯菡萏泬寥窅冥毰毸濩落霅霅便嬛岧峣瀺灂姽婳愔嫕
	飒纚棽俪緸冤莩甲摛藻卮言倥侗椒觞期颐夜阑彬蔚倥偬澄廓簪缨陟遐迤逦缥缃鹣鲽憯懔
	闺闼璀错媕婀噌吰澒洞阛闠覼缕玓瓑逡巡諓諓琭琭瀌瀌踽踽叆叇氤氲瓠犀流眄蹀躞赟嬛
	茕頔璎珞螓首蘅皋惏悷缱绻昶皴皱颟顸愀然菡萏卑陬纯懿犇麤掱暒墌墍墎墏墐墒墒墓墔
	墕墖墘墖墚墛坠墝增墠墡墢墣墤墥墦墧墨墩墪樽墬墭堕墯墰墱墲坟墴墵垯墷墸墹墺墙墼
	墽垦墿壀壁壂壃壄壅壆坛壈壉壊垱壌壍埙壏壐壑壒压壔壕壖壗垒圹垆壛壜壝垄壠壡坜壣
	壤壥壦壧壨坝塆圭嫶嫷嫸嫹嫺娴嫼嫽嫾婳妫嬁嬂嬃嬄嬅嬆嬇娆嬉嬊娇嬍嬎嬏嬐嬑嬒嬓嬔
	嬕嬖嬗嬘嫱嬚嬛嬜嬞嬟嬠嫒嬢嬣嬥嬦嬧嬨嬩嫔嬫嬬奶嬬嬮嬯婴嬱嬲嬳嬴嬵嬶嬷婶嬹嬺嬻
	嬼嬽嬾嬿孀孁孂娘孄孅孆孇孆孈孉孊娈孋孊孍孎孏嫫婿媚嵭嵮嵯嵰嵱嵲嵳嵴嵵嵶嵷嵸嵹
	嵺嵻嵼嵽嵾嵿嶀嵝嶂嶃崭嶅嶆岖嶈嶉嶊嶋嶌嶍嶎嶏嶐嶑嶒嶓嵚嶕嶖嶘嶙嶚嶛嶜嶝嶞嶟峤
	嶡峣嶣嶤嶥嶦峄峃嶩嶪嶫嶬嶭崄嶯嶰嶱嶲嶳岙嶵嶶嶷嵘嶹岭嶻屿岳帋巀巁巂巃巄巅巆巇
	巈巉巊岿巌巍巎巏巐巑峦巓巅巕岩巗巘巙巚帠帡帢帣帤帨帩帪帬帯帰帱帲帴帵帷帹帺帻
	帼帽帾帿幁幂帏幄幅幆幇幈幉幊幋幌幍幎幏幐幑幒幓幖幙幚幛幜幝幞帜幠幡幢幤幥幦幧
	幨幩幪幭幮幯幰幱庍庎庑庖庘庛庝庠庡庢庣庤庥庨庩庪庬庮庯庰庱庲庳庴庵庹庺庻庼庽
	庿廀厕廃厩廅廆廇廋廌廍庼廏廐廑廒廔廕廖廗廘廙廛廜廞庑廤廥廦廧廨廭廮廯廰痈廲廵
	廸廹廻廼廽廿弁弅弆弇弉弖弙弚弜弝弞弡弢弣弤弨弩弪弫弬弭弮弰弲弪弴弶弸弻弼弽弿
	彖彗彘彚彛彜彝彞彟彴彵彶彷彸役彺彻彽彾佛徂徃徆徇徉后徍徎徏径徒従徔徕徖徙徚徛
	徜徝从徟徕御徢徣徤徥徦徧徨复循徫旁徭微徯徰徱徲徳徴徵徶德徸彻徺忁忂惔愔忇忈忉
	忔忕忖忚忛応忝忞忟忪挣挦挧挨挩挪挫挬挭挮挰掇授掉掊掋掍掎掐掑排掓掔掕挜掚挂掜
	掝掞掟掠采探掣掤掦措掫掬掭掮掯掰掱掲掳掴掵掶掸掹掺掻掼掽掾掿拣揁揂揃揅揄揆揇
	揈揉揊揋揌揍揎揑揓揔揕揖揗揘揙揤揥揦揧揨揫捂揰揱揲揳援揵揶揷揸揻揼揾揿搀搁搂
	搃搄搅搇搈搉搊搋搌搎搏搐搑搒摓摔摕摖摗摙摚摛掼摝摞摠摡斫斩斮斱斲斳斴斵斶斸旪
	旫旮旯晒晓晔晕晖晗晘晙晛晜晞晟晠晡晰晣晤晥晦晧晪晫晬晭晰晱晲晳晴晵晷晸晹晻晼
	晽晾晿暀暁暂暃暄暅暆暇晕晖暊暋暌暍暎暏暐暑暒暓暔暕暖暗旸暙暚暛暜暝暞暟暠暡暣
	暤暥暦暧暨暩暪暬暭暮暯暰昵暲暳暴暵暶暷暸暹暺暻暼暽暾暿曀曁曂曃晔曅曈曊曋曌曍
	曎曏曐曑曒曓曔曕曗曘曙曚曛曜曝曞曟旷曡曢曣曤曥曦曧昽曩曪曫晒曭曮曯椗椘椙椚椛
	検椝椞椟椠椡椢椣椤椥椦椧椨椩椪椫椬椭椮椯椰椱椲椳椴椵椶椷椸椹椺椻椼椽椾椿楀楁
	楂楃楅楆楇楈楉杨楋楌楍榴榵榶榷榸榹榺榻榼榽榾桤槀槁槂盘槄槅槆槇槈槉槊构槌枪槎
	槏槐槑槒杠槔槕槖槗滙滛滜滝滞滟滠滢滣滦滧滪滫沪滭滮滰滱渗滳滵滶滹滺浐滼滽漀漃
	漄漅漈漉溇漋漌漍漎漐漑澙熹漗漘漙沤漛漜漝漞漟漡漤漥漦漧漨漪渍漭漮漯漰漱漳漴溆
	漶漷漹漺漻漼漽漾浆潀颍潂潃潄潅潆潇潈潉潊潋潌潍潎潏潐潒潓洁潕潖潗潘沩潚潜潝潞
	潟潠潡潢潣润潥潦潧潨潩潪潫潬潭浔溃潱潲潳潴潵潶滗潸潹潺潻潼潽潾涠澁澄澃澅浇涝
	澈澉澊澋澌澍澎澏湃澐澑澒澓澔澕澖涧澘澙澚澛澜澝澞澟渑澢澣泽浍澯澰淀澲澳澴澵澶
	澷澸潇潆瀡瀢瀣瀤瀥潴泷濑瀩瀪瀫瀬瀭瀮瀯弥瀱潋瀳瀴瀵瀶瀷瀸瀹瀺瀻瀼瀽澜瀿灀灁瀺
	灂沣滠灅灆灇灈灉灊灋灌灍灎灏灐洒灒灓漓灖灗滩灙灚灛灜灏灞灟灠灡灢湾滦灥灦灧灨
	灪燝燞燠燡燢燣燤燥灿燧燨燩燪燫燮燯燰燱燲燳烩燵燵燸燹燺薰燽焘燿爀爁爂爃爄爅爇
	爈爉爊爋爌烁爎爏爑爒爓爔爕爖爗爘爙爚烂爜爝爞爟爠爡爢爣爤爥爦爧爨爩猽猾獀犸獂
	獆獇獈獉獊獋獌獍獏獐獑獒獓獔獕獖獗獘獙獚獛獜獝獞獟獠獡獢獣獤獥獦獧獩狯猃獬獭
	狝獯狞獱獳獴獶獹獽獾獿猡玁玂玃。

	\begin{leftbar}
		\noindent\textbf{建议:}附录为可选部分,一般用来存放不适合在正文中列出的
		大段内容,如较长的表格、代码段等。若需分小节列出,则应采用加星号版本的小
		节命令\verb|\section*{}|,以避免出现不正确的章节号。
	\end{leftbar}

\end{appendices}
\end{document}
%---------------------------------------------------------------------------%

